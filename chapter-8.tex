\section{Eighth session: Zeros \& Poles}

In this chapter, we study the behavior of a holomorphic function $f$ around a point $c$ that may or may not be in its domain of definition. These considerations are interesting because they give way to important mathematical arguments like the maximum principle.

We'll limit ourselves to isolated singularities. Recall:
\begin{defi}[Recall: Isolated Singularity]
    Let $f: \Omega \rightarrow \mathbb{C}$ be a holomorphic function where $\Omega$ is an open subset of $\mathbb{C}$.
    
    $a\in\mathbb{C}\setminus\Omega$ is a singularity of $f$ if
    $$ \exists \varepsilon > 0 \vert \forall z\in\mathbb{C}, z \neq a, |z-a| < \varepsilon \implies z\in\Omega$$
\end{defi}

\begin{note}
    It can be useful to remark that a point $a$ is isolated in a closed set $C$ is $C\setminus \{a\}$ is still closed. 

    Indeed, the set $\mathbb{C}\setminus\Omega$ is closed, as the complement of an open set.
\end{note}

\begin{note}
    We can summarize the definition of an isolated singularity by requiring that an annulus $A(c, 0, r)$ with $r>0$ is a subset of $\Omega$.
\end{note}


First, we'll look at the zeros of a holomorphic function. Then, we'll study the poles. Finally, we'll show we can use line integrals (residues, in fact) to study zeros and poles.


\subsection{Zeroes of a holomorphic function}
\subsubsection{Multiplicity}
\begin{defi}
    [Zero]
    Let $\Omega$ be an open subset of $\mathbb{C}$.

    Let $f: \Omega \rightarrow \mathbb{C}$.

    A zero (or root) $c$ of $f$ is a point $c\in\Omega$ such that $$ f(c) = 0$$
\end{defi}

This vocabulary is the same as for polynomials. Recall that polynomial roots can have multiplicities (e.g. $(X-1)^3$). We saw in the previous section that holomorphic functions "are like" polynomials if we expand them in a Taylor series, thus:

\begin{defi}
    [Zero]
    Let $\Omega$ be an open subset of $\mathbb{C}$.

    Let $f: \Omega \rightarrow \mathbb{C}$.

    A zero (or root) $c\in\Omega$ of $f$ is of multiplicity $p$ if there exists $a\in\mathbb{C}$ such that
    $$ f(z) \sim_{c}a(z-c)^p $$
    
    Equivalently, if 

    $$\lim_{z\rightarrow c} \frac{f(z)}{(z-c)^p} =a \in\mathbb{C}$$

    A pole of multiplicity $ 1$ is simple, $2$ is double, \dots 
\end{defi}


The folowing theorem characterizes the multiplicity of a zero:
\begin{thm*}
    [Characterisation of the multiplicity of a zero]
    Let $\Omega$ be an open subset of $\mathbb{C}$.

    Let $f: \Omega \rightarrow \mathbb{C}$ be a holomorphic function.

    A zero $c\in\Omega$ of $f$ is of multiplicity $p$ if and only if one of the following equivalent conditions holds:

    \begin{itemize}
        \item The function $f$ and exactly its first $p-1$ derivatives are zero at $c$, i.e. $\forall k < p, f^{(k)}(c) = 0$ but $f^{(p)}(c) \neq 0$.

        \item The Taylor expansion of $f$ at $c$ is 

            $$ f(z) = \sum_{n=p}^{+\infty}(z-c)^n$$ with $a_p\neq 0$.

        \item There is a holomorphic function $a$ such that 

            $$ \forall z\in\Omega, f(z) = a(z) (z-c)^p$$
            with $a(c) \neq 0$.
    \end{itemize}


\end{thm*}

The intuition behind holomorphic functions is their proximity with polynomials (or rational functions). The following theorem should not come as a surprise then:

\begin{thm*}
    [Zero with no finite multiplicity]
    Let $\Omega$ be an open, connected subset of $\mathbb{C}$.

    Let $f: \Omega \rightarrow \mathbb{C}$ be a holomorphic function.

    If $c$ is a zero of $f$ but has no finite multiplicity, then $f$ is identically zero.
\end{thm*}

\subsubsection{Isolated zeros}

Conversely, a zero with finite multiplicity is isolated:

\begin{thm*}
    [Zeros of finite multiplicity are isolated]

    Let $\Omega$ be an open subset of $\mathbb{C}$.

    Let $f: \Omega \rightarrow \mathbb{C}$ be a holomorphic function.

    If $c$ is a zero of $f$ with finite multiplicity, then $c$ is isolated in $\Omega$.
\end{thm*}

From this we'll deduce that:

\begin{thm*}
    [Isolated Zeros Theorem I]
    Let $\Omega$ be an open, connected subset of $\mathbb{C}$.

    Let $f: \Omega \rightarrow \mathbb{C}$ be a holomorphic function.
    
    Unless $f$ is identically 0, the zeros of $f$ are isolated.
\end{thm*}

\begin{exo}
    Can you prove the previous theorem using properties we just saw?
\end{exo}

\begin{note}
    [In practice]
    
    This theorem is mostly useful for showing that a holomorphic function is identically zero, using its contraposition.
\end{note}

To explicit this, let's formalize the opposite of a set with an isolated point: a limit point.

\begin{defi}
    [Limit point]
    A point $c$ is a limit point of $C\subset \mathbb{C}$ if every open annulus $A(c, 0, r)$ intersects $C$, i.e.

    $$ \forall r > 0, A(c, 0, r) \cap C \neq \emptyset $$
\end{defi}

We can now use:

\begin{thm*}
    [Isolated Zeros Theorem II]
    Let $\Omega$ be an open, connected subset of $\mathbb{C}$.

    Let $f: \Omega \rightarrow \mathbb{C}$ be a holomorphic function.
    
    If the set of zeros of $f$ has a limit point, then $f$ is identically zero.
\end{thm*}

A useful corollary is:

\begin{thm*}
    [Uniqueness principle]
    Let $\Omega$ be an open, connected subset of $\mathbb{C}$.

    Let $f_1, f_2: \Omega \rightarrow \mathbb{C}$ be two holomorphic functions.

    If $f_1$ and $f_2$ are equal on a subset of $\Omega$ with a limit point, then $f_1=f_2$ on $\Omega$.
\end{thm*}

\begin{thm*}
    [Permanence principle]
    Let $\Omega$ be an open, connected subset of $\mathbb{C}$.

    Let $f_1, \dots, f_n: \Omega \rightarrow \mathbb{C}$ be $n$ holomorphic functions.

    Let $F : \mathbb{C}^n \mapsto \mathbb{C}$ complex-differentiable on a subset of $\mathbb{C}^n$.

    If the set of points such that $F(f_1(z), \dots, f_n(z)) = 0$ has a limit point in $\Omega$, then $$\forall z\in\Omega, F(f_1(z), \dots, f_n(z))=0$$
\end{thm*}

\begin{exo}
    [$\star$]
    Show that the triangular identity $\sin^2 x + \cos^2 x = 1$ extends to the complex plane, in tw different ways.
\end{exo}

\subsection{Isolated Singularities of a holomorphic function}

\begin{defi}
    [Typology of isolated singularities]

    Let $\Omega$ be an open subset of $\mathbb{C}$.

    Let $f: \Omega \rightarrow \mathbb{C}$ be a holomorphic function.

    An isolated singularity $c$ of $f$ is:

    \begin{itemize}
        \item a removable singularity if there is a holomorphic extension of $f$ over $c$. In other words, there is $a: \Omega\cup\lbrace c \rbrace$ holomorphic such that $\forall z \in\Omega, f(z) = a(z) $.
        \item a pole of multiplicity $p$ for some $p \in \mathbb{N}^*$ if there is an $a\in \mathbb{C}^*$ such that:
            $$ f(z) \sim \frac{a^*}{(z-c)^p}$$ or equivalently $$f(z)(z-c)^p \sim a^*$$
        \item an essential singularity otherwise.
    \end{itemize}
\end{defi}

First let's characterize removable singularities.

\begin{thm*}
    [Characterization of Removable Singularities]

    An isolated singularity $c$ of a holomorphic function $f: \Omega \rightarrow \mathbb{C}$ is removable if and only if one of the following conditions holds:

    \begin{enumerate}
        \item The Laurent expansion of $f$ is some non-empty annulus $A(c, 0, r)$ is a power series (its coefficients $a_n$ are zero for $n<0$).
        \item The value $f(z)$ has a limit in $\mathbb{C}$ when $z\rightarrow c$.
        \item The function $f$ is bounded in some non-empty open annulus $A(c, 0, r)$.
    \end{enumerate}
\end{thm*}

\begin{proof}
    We can prove $ (3) \implies (1)$ using the expression of $a_n$.
\end{proof}

\begin{thm*}
    [Characterization of Pole Multiplicity]

    An isolated singularity $c$ of a holomorphic function $f: \Omega \rightarrow \mathbb{C}$ is a pole of multiplicity $p$ if and only if one of the following conditions holds:
    \begin{enumerate}
        \item The Laurent expansion of $f$ is some non-empty annulus $A(c, 0, r)$ is 
            $$ f(z) = \sum_{n=-p}^{+\infty} a_n(z-c)^n$$
            with $ a_{-p} \neq 0$.
        \item There is a holomorphic function $a: \Omega \rightarrow \mathbb{C}$ such that 
            $$\forall z \in \Omega, f(z) = \frac{a(z)}{(z-c)^p} $$
    \end{enumerate}
\end{thm*}

\begin{thm*}
    [Characterization of Poles]

    An isolated singularity $c$ of a holomorphic function $f: \Omega \rightarrow \mathbb{C}$ is a pole (of multiplicity $p$) if and only if the following conditions hold:
    \begin{enumerate}
        \item $ 1/f $ is defined in some open annulus $A(c, 0, r)$,
        \item $ 1/f$ has a holomorphic extension to $D(c, r)$
        \item and $c$ is a zero (of multiplicity $p$) of this extension.
    \end{enumerate}

    Alternatively, $c$ is a pole of $f$ if and only if
    $$ |f(z)| \rightarrow + \infty$$ when $$ z \rightarrow c $$
\end{thm*}


\subsection{Computation of residues}

We can use residues to characterize the behaviour of $f$ around its singularities.

\begin{thm*}
    [Computation of residues]
    Let $\Omega$ be an open subset of $\mathbb{C}$.

    Let $f: \Omega \rightarrow \mathbb{C}$ be a holomorphic function.

    Let $c$ an isolated singularity of $f$.

    If the Laurent series expansion of $f$ in some non-empty annulus $A(c, 0, r)\subset Omega$ is 

    $$ f(z) = \sum_{n=-\infty}^{+\infty} a_n(z-c)^n $$

    then $$ \res(f, c) = a_{-1} $$
    
\end{thm*}

\begin{proof}
    Calculate formally then justify the series-integral inversion.
\end{proof}

\begin{thm*}
    [Residue of Poles]
    Let $\Omega$ be an open subset of $\mathbb{C}$.

    Let $f: \Omega \rightarrow \mathbb{C}$ be a holomorphic function.

    Let $c$ an isolated singularity of $f$.

    If $c$ is a pole of $f$ whose multiplicity is at most $p$:

    $$\exists a\in\mathbb{C}^*, \lim_{z\rightarrow c}f(z)(z-c)^p = a$$

    then 
    
    $$ \res(f, c) = \lim_{z\rightarrow c}{ \frac{1}{(p-1)!}\frac{d^{p-1}}{dz^{p-1}}(f(z)(z-c)^p) }$$
\end{thm*}


In particular we have the following results:

\begin{thm*}
    [Residue of Simple Poles I]
    Let $\Omega$ be an open subset of $\mathbb{C}$.

    Let $f: \Omega \rightarrow \mathbb{C}$ be a holomorphic function.

    Let $c$ an isolated singularity of $f$.

    The point $c$ is a simple pole of $f$ if and only if
    
    $$\exists a\in\mathbb{C}^*, \lim_{z\rightarrow c}f(z)(z-c) = a$$

    and then
    
    $$ \res(f, c) = \lim_{z\rightarrow c}{f(z)(z-c)}$$
\end{thm*}

\begin{thm*}
    [Residue of Simple Poles II]
    Let $\Omega$ be an open subset of $\mathbb{C}$.

    Let $f: \Omega \rightarrow \mathbb{C}$ be a holomorphic function.

    Let $c$ an isolated singularity of $f$.

    If there are two holomorphic functions $g$ and $h$ on $\Omega$ such that 

    $$f = \frac{g}{h}$$

    where $g(c) \neq 0$, $h(c) = 0$, and $h'(c) \neq 0$, then c is a simple pole of $f$ and 

    $$ \res(f, c) = \frac{g(c)}{h'(c)}$$
\end{thm*}

