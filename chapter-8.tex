\section{Eighth session: Zeros \& Poles}

In this chapter, we study the behavior of a holomorphic function $f$ around a point $c$ that may or may not be in its domain of definition. These considerations are interesting because they give way to important mathematical arguments like the maximum principle.

We'll limit ourselves to isolated singularities. Recall:
\begin{defi}[Recall: Isolated Singularity]
    Let $f: \Omega \rightarrow \mathbb{C}$ be a holomorphic function where $\Omega$ is an open subset of $\mathbb{C}$.
    
    $a\in\mathbb{C}\setminus\Omega$ is a singularity of $f$ if
    $$ \exists \varepsilon > 0 \vert \forall z\in\mathbb{C}, z \neq a, |z-a| < \varepsilon \implies z\in\Omega$$
\end{defi}

\begin{note}
    It can be useful to remark that a point $a$ is isolated in a closed set $C$ is $C\setminus \{a\}$ is still closed. 

    Indeed, the set $\mathbb{C}\setminus\Omega$ is closed, as the complement of an open set.
\end{note}

\begin{note}
    We can summarize the definition of an isolated singularity by requiring that an annulus $A(c, 0, r)$ with $r>0$ is a subset of $\Omega$.
\end{note}


First, we'll look at the zeros of a holomorphic function. Then, we'll study the poles. Finally, we'll show we can use line integrals (residues, in fact) to study zeros and poles.


\subsection{Zeroes of a holomorphic function}

\begin{defi}
    [Zero]
    Let $\Omega$ be an open subset of $\mathbb{C}$.

    Let $f: \Omega \rightarrow \mathbb{C}$.

    A zero (or root) $c$ of $f$ is a point $c\in\Omega$ such that $$ f(c) = 0$$
\end{defi}

This vocabulary is the same as for polynomials. Recall that polynomial roots can have multiplicities (e.g. $(X-1)^3$). We saw in the previous section that holomorphic functions "are like" polynomials if we expand them in a Taylor series, thus:

\begin{defi}
    [Zero]
    Let $\Omega$ be an open subset of $\mathbb{C}$.

    Let $f: \Omega \rightarrow \mathbb{C}$.

    A zero (or root) $c\in\Omega$ of $f$ is of multiplicity $p$ if there exists $a\in\mathbb{C}$ such that
    $$ f(z) \sim_{c}a(z-c)^p $$
    
    Equivalently, if 

    $$\lim_{z\rightarrow c} \frac{f(z)}{(z-c)^p} =a \in\mathbb{C}$$

    A pole of multiplicity $ 1$ is simple, $2$ is double, \dots 
\end{defi}


The folowing theorem characterizes the multiplicity of a zero:
\begin{thm*}
    [Characterisation of zero multiplicity]
    Let $\Omega$ be an open subset of $\mathbb{C}$.

    Let $f: \Omega \rightarrow \mathbb{C}$ be a holomorphic function.

    A zero $c\in\Omega$ of $f$ is of multiplicity $p$ if and only if one of the following equivalent conditions holds:

    \begin{itemize}
        \item The function $f$ and exactly its first $p-1$ derivatives are zero at $c$, i.e. $\forall k < p, f^{(k)}(c) = 0$ but $f^{(p)}(c) \neq 0$.

        \item The Taylor expansion of $f$ at $c$ is 

            $$ f(z) = \sum_{n=p}^{+\infty}(z-c)^n$$ with $a_p\neq 0$.

        \item There is a holomorphic function $a$ such that 

            $$ \forall z\in\Omega, f(z) = a(z) (z-c)^p$$
            with $a(c) \neq 0$.
    \end{itemize}


\end{thm*}

