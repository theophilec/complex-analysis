\section{Seventh session: Power series}

A power series is the generalization of the polynomial to an infinity of terms, written:
$$\sum_{n=0}^{+\infty}a_n(z-c)^n$$

\subsection{Definition \& Fundamental properties}

\begin{defi}
    The radius of convergence for the above power series is the unique $r$ such that:
    \begin{itemize}
        \item the series converges if $|x - c| < r$ 
        \item the series diverges is $|x - c| > r$.
    \end{itemize}

    In other words, it is the inverse of the growth ratio $\sigma$ of the series, such that:
    $$\exists m \in \mathbb{N}, \forall n \in \mathbb{N} \vert n \geq m \implies |a_n| \leq \sigma^n $$
\end{defi}

In pratice to calculate the radius of convergence, find the smallest $\sigma > 0$ such that "apcr" $$|a_n|^n \leq \sigma^n$$

\begin{note}
    Careful, at the radius of convergence, anything can happen!
\end{note}

\begin{example}
    [$\star$]
    $a_n = 2^n$, $a_n = \left(\frac{1}{2}\right)^n$, \dots
\end{example}

\begin{thm*}
    [Multiplication of power series]
    $$R(a_nb_n) \geq R(a_n)R(b_n)$$

    Equality for polynomial coefficients, i.e. $a_n \sim n^p$.
\end{thm*}
\begin{proof}
    $\sigma_{ab} \leq \sigma_a\sigma_b$
\end{proof}

\subsection{Convergence}
Let's take a look at what happens within a power series' disk of convergence.

\begin{thm*}
    [Locally Normal Convergence]
    The convergence of the power series above in its open disk of convergence $D(c, r)$ is locally normal:
    
    for any $z\in D(c,r)$, there exists an open neighborhood $U$ of $z$ in $D(c, r)$ such that

    $$\exists \kappa > 0, \forall z \in U, \sum_{n=0}^{+\infty}|a_n(z-c)^n| \leq \kappa$$


    Equivalently, for every compact subset $K$ of $D(c,r)$,
    $$\exists \kappa > 0, \forall z \in K, \sum_{n=0}^{+\infty}|a_n(z-c)^n| \leq \kappa$$

\end{thm*}

\begin{note}
    Notice that the convergence properties are on open neighborhoods, or on compact subsets of an open disk. Thus, at the border, nothing is known!
\end{note}
    
\begin{thm*}
    [Other types of convergence]
    Locally normal convergence implies two other types of convergence of the series:

    \begin{description}
        \item[Absolute convergence] 
        $$\forall z \in D(c,r), \sum_{n=0}{+\infty}|a_n(z-c)^n| < \infty$$
        \item[Locally uniform convergence] 
            $$\left\|\sum_{n=0}^{p}a_n(z-c)^n - \sum_{n=0}^{+\infty}an(z-c)^n\right\|\rightarrow_{p\rightarrow\infty} 0$$
    \end{description}
\end{thm*}

\subsection{Derivative, Holomorphy}

Let's take a look at what happens when we differentiate a power series. First, we'll "cheat" and show that the intuitive "formal" derivative has interesting properties.
\begin{thm*}
    [Power series derivative]
    A power series 
$$\sum_{n=0}^{+\infty}a_n(z-c)^n$$
and its formal derivative 
$$\sum_{n=1}^{+\infty}na_n(z-c)^{n-1}$$

have the same radius of convergence.

Furthermore, the sum of a power series is holomorphic on its open disk of convergence.

Its derivative is the sum of its formal derivative on its open disk of convergence.

\end{thm*}

\begin{exo}
    Compute the $p$-th formal derivative of a formal power series.
\end{exo}

\subsection{Taylor series expansion}
We've just seen that a power series is a holomorphic function. What about the converse statement? Power series expansion is the computation of a holomorphic function as a power series, for example as a Taylor Series.


\begin{thm*}[Unicity of the power series expansion -- Taylor series]
    If $f : \mathbb{C} \rightarrow \mathbb{C}$ has a power series expansion centered at $c$ inside the non-empty open disk $D(c,r)$, it is the Taylor series of $f$:
    $$\forall z \in D(c,r), f(z) = \sum_{n=0}^{+\infty}\frac{f^{(n)}(c)}{n!}(z-c)^n$$
\end{thm*}

Here we've shown uniqueness, not existence. Let's do so here:

\begin{thm*}
    [Power series expansion for holomorphic functions]
    Let $\Omega$ be an open subset of $\mathbb{C}$.

    Let $f: \Omega \rightarrow \mathbb{C}$ be a holomorphic function.
    
    Let $c\in\Omega$ and $r\in]0, +\infty]$, such that $D(c,r)\in\Omega$.

    There is a power series with coefficients $a_n$ such that:

    $$ \forall z \in D(c, r), f(z) = \sum_{n=0}^{+\infty}a_n(z-c)^n$$

Its coefficients are given by:

$$ \forall 0 < \rho < r, 
a_n = \frac{1}{2i\pi} \int_\gamma \frac{f(z)}{{(z-c)^{n+1}}}dz$$

where 
$$ \gamma = c + \rho[C_+]$$
\end{thm*}

\subsection{Laurent series}
Taylor series expansions are polynomial. For example, they are exact expansions for polynomial functions. To capture more properties of a function, especially around singularities, we can generalize them using rational fractions, i.e. negative powers of $z$.

\begin{defi}
    [Laurent series]
    A Laurent series centered on $c\in\mathbb{C}$ with coefficients $(a_n)_{n\in\mathbb{Z}}$ is 
    $$ \sum_{n=-\infty}^{+\infty}a_n(z-c)^n$$
\end{defi}

\begin{defi}
    [Laurent series--convergence]
    A Laurent series is convergent at $z\in\mathbb{C}$ if sums
    $$ \sum_{n=0}^{+\infty}a_n(z-c)^n$$
    and
    $$ \sum_{n=1}^{+\infty}a_{-n}(z-c)^{-n}$$
    are both convergent.
\end{defi}

For obvious reasons ($z-c$ can be $= 0$), we want to define the Laurent series excluding some interior points instead of defining it on an open disk. Let's introduce the annulus for this.

\begin{defi}
    [Annulus]
    Let $c\in\mathbb{C}$ and $r_1, r_2 \in [0, + \infty]$.

    We denote by 
    $$A(c, r_1, r_2) = \left\lbrace z\in\mathbb{C} \vert r_1 < | z-c | < r_2 \right \rbrace $$
\end{defi}
\begin{exo}
    Show that an annulus is open.
\end{exo}

Let's provide some of the same results as for Taylor series.

\begin{thm*}
    [Convergence of Laurent Series ]
    The inner radius of convergence 
    $$r_1  = \limsup_{n\rightarrow + \infty}|a_{-n}|^{1/n} $$
    and the outer radius of convergence
    $$r_2  = \frac{1}{\limsup_{n\rightarrow + \infty}|a_{-n}|^{1/n}} $$

    are such that the series converges in $A(c, r_1, r_2)$ and diverges if $|z-c| < r_1$ or $|z-c| > r_2$.

    In this open annulus of convergence, the convergence is locally normal.
\end{thm*}

\begin{note}
    Like for Taylor series, the annulus of convergence is open! So what happens at the boundary is not well characterized in general.
\end{note}

Finally, 
\begin{thm*}
    [Laurent series expansion for holomorphic functions]
    Let $\Omega$ be an open subset of $\mathbb{C}$.

    Let $f: \Omega \rightarrow \mathbb{C}$ be a holomorphic function.
    
    Let $c\in\Omega$ and $r_1, r_2\in[0, +\infty]$, such that $A(c,r_1, r_2)\in\Omega$ is non-empty.

    There is a Laurent series with coefficients $a_n$ such that:

    $$ \forall z \in A(c, r_1, r_2), f(z) = \sum_{n=-\infty}^{+\infty}a_n(z-c)^n$$

Its coefficients are given by:

$$ \forall r_1 < \rho < r_2, 
a_n = \frac{1}{2i\pi} \int_\gamma \frac{f(z)}{{(z-c)^{n+1}}}dz$$

where 
$$ \gamma = c + \rho[C_+]$$
\end{thm*}



