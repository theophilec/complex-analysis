\section*{Introduction}
\subsection*{C1123 Introduction}
Welcome to C1223! I hope you'll enjoy this very interesting and unique course at Mines ParisTech.

It is a great opportunity to learn about an important field in mathematics (including Applied Mathematics) and one of the last opportunities for you to stretch your ``fundamental" math skills before you become engineers.

Whether you've appreciated math courses in the past or not, there should be something in this course for you. The theory is both intuitive (we'll be drawing a lot) and beautiful (things are simple).

\subsubsection*{Pre-requisites}
There are relatively few prerequisites for this course apart from basic linear algebra (vector spaces, ...), basic topology (open and closed sets, ...) and analysis (Taylor series, ...).

The textbook is mostly self-contained, and these notes present the background you may be lacking (or have forgotten).

\subsubsection*{Methodological reminders}
The ``lectures" for C1223 last an hour and are directly followed by a problem sesssion (with another instructor). There are two such sessions per day for a week so the course is in fact quite intense.

Some advice:
\begin{description}
    \item[Stay focused!] The course is fast-paced and you won't have much time to absorb the material between sessions! However, rest assured that the material is tailor-made to fit to this format. Just go with the flow and the ideas will progressively mature over the week and following weeks before the exam. I'll try to provide context as we go. 
    \item [Ask questions!] The goal is for everyone to be able to make progress and asking questions definitely contributes to this. As soon as the first person asks his or hers, I'm confident the rest will follow!
    \item [Draw!] In Complex Analysis, a picture by proof is often more than half way to a solution. In fact, without drawings some parts of the material would be virtually impossible to present.  Although some of the proofs are technically challenging, here, we are interested in giving you the intuition necessary for understanding and using the tools presented.
    \item [Finally, give feedback!] This is a learning experience for me at least as much as it is for you. If something is unclear or poorly explained, ask questions! And if the pace is too slow or too fast, speak up!
\end{description}

\subsubsection*{About this document}
Before we start, a few words about this document:

First, what is it not?
\begin{description}
    \item [A replacement textbook] S.B.'s text is extremely well written and we'll be using it as the main tool for the course. It contains the material, exercices and their solutions.
    \item [A cheat sheet] You should be getting one of those at the end of the course to help in your review before the exams.
\end{description}
What is it then?
\begin{description}
    \item [A summary] I try to present the most important elements of the text, without any proofs but with illustrations, examples, counter-examples and heuristics.
    \item [A study-guide] If you can master everything that is here, you should be fine.
    \item [My notes] I wrote this while preparing for TA'ing this course, so actually, it is mostly for me. I hope it will be useful for you too though.
\end{description}

A final note (and disclaimer): these pages are not a subset of the textbook material nor are they a superset... The syllabus will be explicited by the Professors responsible for the course. Nothing here is official nor endorsed by them. The material I've added are mostly reminders from prerequisites and extra counter-examples... basically, what I had scribbled in the margin of my textbook when I took the course.
