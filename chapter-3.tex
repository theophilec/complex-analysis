\section{Third Session: Connected sets}
In this section, we characterize subsets of the complex plane (we refer to them simply as sets) that are "in one piece".

We present two alternative concepts: path-connectedness and connectedness. The latter is more abstract but more powerful. When the sets we consider are open, the two properties are equivalent.

\subsection*{Contents}
\begin{enumerate}
    \item \dots
        \begin{enumerate}
            \item \dots
        \end{enumerate}
\end{enumerate}

\subsection{Path-connected, connected}

\subsubsection{Path-connected}
We've already seen and used the definition of path connected but as a reminder:
\begin{defi}[Path-connected set]
   $A$ is path connected if any two points of $A$ can be joined by a path on $A$:

   $$\forall a,b\in A, \exists \gamma\in\mathcal{C}^0, \forall t\in[0,1], \gamma(t)\in A~\mathrm{ and}~\gamma(0) = a, \gamma(1) = b$$
\end{defi}

\subsubsection{Connected}
In order to define connectedness, let's introduce the concept of \emph{dilation}.

\begin{defi}[Dilation]
    A set $B$ is a dilation of $A$ if it the union of a collection of non-empty open disks whose centers are at the points of A:

    $$ B = \bigcup_{a\in A}{D(a, r_a)} $$
    where $\forall a\in A, r_a > 0$.
\end{defi}

We can now define connectedness:

\begin{defi}
    A set is connected if all of its dilations are path connected. A set that does not verify this property is disconnected.
\end{defi}

\begin{thm*}[Path-connected $\implies$ connected]
    Every path-connected set is connected.
\end{thm*}

\begin{proof}[Proof sketch $(\implies)$]
    Take $z, w \in B$ a dilation of $A$. There is $a, b \in A$ such that $z\in D_a$ and $w\in D_b$.
    As in the drawing, there is a path from $z$ to $a$ in $D_a$, and path from $b$ to $w$. Because $A$ is path-connected, there is a path from $a$ to $b$ in $A$. 
\end{proof}

\begin{thm*}[Open, connected $\implies$ path-connected]
    Every open, connected set is path-connected.
\end{thm*}

Let's quickly prove this theorem with a drawing:

\begin{proof}[Proof sketch $(\impliedby)$, not as pedagogical]
    Show that it is legitimate to write $A$ as a dilation of itself.
\end{proof}

Finally, we can deduce the following theorem:

\begin{thm*}[Open connected $\iff$ open, path-connected]
   An open set is connected if and only if it is path-connected. 
\end{thm*}

\subsubsection{Properties of path-connected/connected sets}

\begin{thm*}
    The following properties are shared by both connected and path-connected sets:
    \begin{description}
        \item[$\cap \mathcal{A} \neq \emptyset \implies \cup \mathcal{A}$ connected: ] if $\mathcal{A}$ is a collection of (path-)conected sets whose intersection $\cap \mathcal{A}$ is non-empty, then the union $\cup \mathcal{A}$ is (path-)connected.
        \item[$A \cap B = \emptyset$, with $A,B$ open $ \implies A\cup B$ disconnected] If $A$ and $B$ are two open, non-empty and disjoint sets, then their union is not (path-)connected.
    \end{description}
\end{thm*}

The following property only holds for connectedness:
\begin{thm*}[Closure of connected sets]
    The closure of a connected set is connected.
\end{thm*}

\begin{note}
    This is \emph{not} true for all path-connected sets. Refer to the textbook for a counter-example.
\end{note}

\subsection{Components}
\begin{defi}
    $B\subset A$ is a (path-)-connected component of $A$ if:

    \begin{itemize}
        \item $B$ is (path-)connected
        \item $B$ is maximal with respect to inclusion (i.e. if $B\subsetneq C\subset A$, then $C=A$)
    \end{itemize}
\end{defi}

\begin{thm*}
    The (path-)connected components of a non-empty set $A$ are a partition of $A$, i.e.:
    \begin{itemize}
        \item they are non-empty
        \item pairwise disjoint 
        \item union is $A$.
    \end{itemize}
\end{thm*}

\begin{thm*}
    A non-empty set is (path-)connected if and only if it has a single (path-)connected component.
\end{thm*}

Because connectedness and path-connectedness are equivalent when a set is open, we can show that:

\begin{thm*}
    The partitions of a non-empty open set into path-connected components and connected components are identical. All such components are open.
\end{thm*}

\begin{note}
    This result signifies that we can use "the" decomposition into (path-)connected components.
\end{note}

\subsection{Locally constant functions}
The following concept will be useful in the future.

\begin{defi}
    A function $f: A \rightarrow \mathbb{C}$ is locally constant if for any $a\in A$ there is a non-empty open disk $D$ centered on $a$ such that $f$ is constant on $A\cap D$.

    In other words, if:
    $$ \forall a \in A, \exists \varepsilon >0, \forall b \in A, |b-a|< \varepsilon \implies f(b) = f(a) $$
\end{defi}

\begin{thm*}
    A set $A$ is connected if and only if every locally constant function defined on $A$ is constant.
\end{thm*}

