\section{Fourth Session: Cauchy's integral theorem}
More so than the other chapters, this chapter is very condensed compared to the textbook. This is mainly because the textbook gives proofs for all the theorems presented.

We will skip the intermediate results and only present the results we need in subsequent chapters. Time-permitting, we can take a look at the different steps of the proof.

\begin{thm*}[Cauchy's integral theorem -- star-shaped version]
    Let $f: \Omega \rightarrow \mathbb{C}$ be a holomorphic function where $\Omega$ is an open star-shaped subset of $\mathbb{C}$.

    For any rectifiable closed path $\gamma$ of $\Omega$:

    $$ \int_\gamma f(z)dz = 0 $$
\end{thm*}

Recall the definition of star-shaped:

\begin{defi}[Star-shaped set]
    A open set $\Omega$ is star-shaped if there exist at least one point $c\in\Omega$ such that for all other points $z$ in $\Omega$, the segment $[c, z]$ is included in $\Omega$.
\end{defi}

\subsection*{Consequences of Cauchy's Integral Theorem}
CIT gives many interesting corollaries that are useful in practice.


\begin{thm*}[Cauchy's Integral Formula for Disks]
    Let $\Omega$ be an open subset of the complex plane. 

    Let $\gamma = c + r[C_+]$ be an oriented circle such that $B_f(c, r) \subset \Omega$.

    For any holomorphic function $f: \Omega \rightarrow \mathbb{C}$, 

    $$\forall z\in\mathcal{B}_f(c, r), f(z) = \frac{1}{2i\pi}\int_\gamma \frac{f(w)}{w - z}dw$$
\end{thm*}


\begin{thm*}
    The derivative of a holomorphic function if holomorphic.
\end{thm*}

\begin{proof}
    You'll prove this theorem during the problem session.
\end{proof}

\begin{note}
    We've already used this property in the past, but rest assured, there are no circular arguments.
\end{note}

\begin{thm*}[Morera's theorem -- characterisation of holomorphy]
    Let $\Omega$ be an open subset of the complex plane. 

    A function $f: \Omega \rightarrow \mathbb{C}$ is holomorphic if and only if:

    \begin{itemize}
        \item it is continuous
        \item locally, its line integrals along rectifiable closed paths are zero.
    \end{itemize}

    In other words, if and only if: for any $c\in\Omega$, there is an $r > 0$ such that $D(c, r) \in \Omega$ and for any rectifiable closed path $\gamma$ of $D(c,r)$,
    $$\int_\gamma f(z)dz = 0 $$
\end{thm*}

\begin{thm*}[Uniform limit of Holomorphic functions]
    Let $\Omega$ be an open subset of the complex plane. 

    If $f_n: \Omega \rightarrow \mathbb{C}$ a sequence of holomorphic functions converges loclly uniformly to a function $f: \Omega \rightarrow \mathbb{C}$, then $f$ is holomorphic.

    In other words, if $\forall c\in\Omega, \exists r > 0, D(c,r) \subset \Omega$ and 

    $$ \lim_{n\rightarrow \infty}\left\| f_n - f \right \|  = 0$$

\end{thm*}

Finally, a last useful theorem:

\begin{thm*}[Liouville's theorem]
    A bounded, holomorphic function defined on $\mathbb{C}$ (\emph{entire}) is constant.
\end{thm*}

Because this theorem is very useful (and to use the previous results) let's prove it step-by-step.

\begin{proof}
    To show that $f$ is constant, we show that $f' = 0$.
    \begin{description}
        \item[First, apply Cauchy's formula for disks to $f'$.]
            We get:
            $$ \forall z\in\mathbb{C}, f'(z) = \frac{1}{2i\pi} \int_\gamma\frac{f'(w)}{w-z}dw$$
            where $\gamma = z + r[C_+]$.
        \item[Second, integrate by parts to make $f$ appear.]
            We get:
            $$ \forall z\in\mathbb{C}, f(z) = \frac{1}{2i\pi} \int_\gamma\frac{f(w)}{(w-z)^2}dw$$
        \item[Third, show that $|f'| < \frac{K}{r}$ for all $r > 0$.] Use the M-L inequality.
        \item[Conclude.] With $r\rightarrow \infty$, $f'(z) = 0$. Because the zero function and $f$ are both primitives of $f'$, they differ by a constant (see Primitives chapter). Thus, $f$ is constant.
    \end{description}
\end{proof}

