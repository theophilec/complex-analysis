\section{First session: Complex-differentiability} 
\subsection*{Contents}
\begin{enumerate}
    \item Reminders
        \begin{enumerate}
            \item topology: open sets (definition and examples)
            \item $\mathbb{C}$, $\mathbb{C}$-vector spaces, $\mathbb{C}$-linearity
            \item (Analysis in $\mathbb{R}$: characterisation of differentiability)
        \end{enumerate}
    \item Complex differentiability:
    \begin{enumerate}
        \item How do we define complex-differentiabiity? 
        \item the complex-derivative? 
        \item the complex-differential? 
        \item What is the difference between real-differentiability and complex-differentiability?
     \end{enumerate}
        
    \item Complex differentiability in practice
    \begin{enumerate}
        \item Usual theorems
        \item Cauchy-Reimann equations: two formulations
    \end{enumerate}
\end{enumerate}

\subsection{Quick reminders}

\subsubsection{Open sets}
\begin{defi}[Open set]
    A set $\Omega \subset A$ is open in $A$ when:$ \forall a\in\Omega, \exists r > 0, \forall y \in \mathcal{B}_r(a), y \in \Omega $
\end{defi}

Alternatively:
    \begin{itemize}
        \item $ \forall a\in\Omega, \exists r > 0, \forall y \in A, ||y - a|| < r \implies y \in \Omega $
        \item $\mathring \Omega = \Omega$
    \end{itemize}
In practice:
\begin{enumerate}
    \item It can be "clear": the goal of this course is not to show that sets are open. 
    \item Write $\Omega$ as the inverse image of an open set by a continuous map. For example, the open unit ball is the inverse image of $]0,1[$ by $x\in E \mapsto ||x||$.
\end{enumerate}
\begin{schema} Visually, this means you can get as close as you want to the complement of the open set $\Omega$ and your neighbors are still in $\Omega$.
\end{schema}

\begin{example}
A few examples of open sets are: $\mathbb{R}$, $\mathbb{C}$, $]0,1[$. $[0,1]$ and $[0,1[$ are not open.
\end{example}

\subsubsection{The complex plane \& complex functions}
Definitions and properties:
\begin{enumerate}
    \item Definition: $\lbrace x+iy, x\in\mathbb{R}, y\in\mathbb{R} \rbrace$, $i$ such that $i^2=1$.
    \item Exponential, trigonometric forms: $$\exists R>0, \theta, x+iy = R(\cos\theta + i\sin\theta)=R\exp(i\theta) $$
    \item Conjugate: $z=x+iy \mapsto \bar z = x - iy$, $z = Re^{i\theta} \mapsto \bar z = Re^{-i\theta}$
    \item \dots 
\end{enumerate}

\begin{exo}
    On the structure of $\mathbb{C}$ as a vector space.
    \begin{enumerate}
        \item Show that the complex plane $\mathbb{C}$ is a complex vector space. 
        \item Give a basis. 
        \item What changes if we consider it to be a real vector space?
    \end{enumerate}
\end{exo}

\begin{exo}
    Give an example of a $\mathbb{C}$-linear function. Give a counter-example (an $\mathbb{R}$-linear function from $\mathbb{C}$ to $\mathbb{C}$ that is not $\mathbb{C}$-linear.)
\end{exo}


\subsection{Complex-differentiable, derivative and differential}
In this section, $\Omega$ is an open subset of $\mathbb{C}$.

\begin{defi}[Complex-differentiability \& Derivative]
    Let $z$ an interior point in $\Omega$. Then, $f$ is \emph{complex-differentiable} at $z$ iff the \emph{complex derivative} of $f$ exists in $\mathbb{C}$, as defined by 
    $$ f'(z) = \lim_{h\rightarrow 0, h\in\mathbb{C}}\frac{f(z+h) - f(z)}{h} $$ 
\end{defi}

\begin{note}
    Let's insist on the $h\in\mathbb{C}$ in the definition. This means that the limit exists no matter the direction from which we are "arriving" in $\mathbb{C}$. Also, no matter the direction, the value is the same.
\end{note}
\begin{note}
    The previous definition is a \emph{local} property. Let's generalize it to $\Omega$ below.
\end{note}

\begin{defi}[Holomorphic]
    $f: \mathbb{C} \rightarrow \mathbb{C}$ is \emph{holomorphic} if $\forall z\in\Omega$, it is \emph{complex-differentiable}, i.e. \emph{complex-differentiable everywhere.}
\end{defi}
 
\begin{example} Some examples of holomorphic functions. Also, a counter-example.
    \begin{itemize}
        \item Some "obvious" examples: constant functions, identity, affine, \dots
        \item ($\star$) Show the inverse function is holomorphic where it is defined. 
        \item ($\star$) A counter example: the complex conjuguate function is nowhere complex-differentiable.
    \end{itemize}
\end{example}

\begin{defi}
    Let $f: A\subset \mathbb{C} \rightarrow\mathbb{C} $.

    Let $z\in A$ as interior point. 

    We define the \emph{complex-differential} (or $\mathbb{C}$-differential) of $f$ in $z$ a complex-linear, continous operator $df_x: \mathbb{C} \rightarrow \mathbb{C} $ that verifies:
        $$ \lim_{h\rightarrow 0, h\in\mathbb{C}}\frac{||f(z+h) - f(z) - df_z(h)||}{||h||} = 0 $$ 
    Equivalently:
        $$ f(z+h) = f(z) + df_z(h) + o(h) $$ 
\end{defi}
\begin{note}
    Recall that $$g(h) = o(h) \iff \lim_{h\rightarrow 0}\frac{g(h)}{h} =0 $$
\end{note}
\begin{note}
    This result extends to $f : E \rightarrow F$ where $E$ and $F$ are complex, normed vector spaces. For example, $\mathbb{C}^n$.
\end{note}


Now, an important theorem that reflects the structure of $\mathbb{C}$:
\begin{thm*}
    Let $f: A \subset\mathbb{C} \rightarrow \mathbb{C}$. Let $z \in \mathring A$. 
    
    The complex-differential of $f$ $df_z$ exists iff the derivative $f'(z)$ exists.
    
    In this case, we have:
    $$\forall h \in \mathbb{C}, df_z(h) = f'(z)h$$
\end{thm*}

\begin{note}
    This is linked to our view of $\mathbb{C}$ as a complex vector space, of dimension $1$.
\end{note}

\subsection{Complex-differentiability in practice}

\subsubsection{Calculus}
\begin{itemize}
    \item $\mathbb{C}$-linear combination
    \item Product
    \item Chain rule
    \item Quotient
    \item Polynomials
    \item Rational fractions
\end{itemize}

In practice: \emph{calculate like in $\mathbb{R}$}.

These "usual" properties allow us to verify that functions are $\mathbb{C}$-differentiable or holomorphic easily, when they are products and compositions (for example).

\subsubsection{Cauchy-Reimann Equations}

We can rephrase the definitions as:
\begin{thm*}
$f$ is complex differentiable on $\Omega$ if and only if both:
\begin{enumerate}
    \item $f$ is real-differentiable on $\Omega$ (i.e. its differential exists, but maybe isn't $\mathbb{C}-linear$)
    \item its real-differential is $\mathbb{C}$-linear.
\end{enumerate}
\end{thm*}

But the second condition is not easy to prove without expliciting the differential. The Cauchy-Riemann equations offer an alternative formulation of this condition.

The intuition behind the theorem is to look at what happens to the real and imaginary parts of $f$ when we move the real and imaginary parts of $z$

If we write:
\begin{enumerate}
    \item if $f: \mathbb{C} \rightarrow \mathbb{C}$, we have $u: \mathbb{C} \rightarrow \mathbb{R}$ and $v: \mathbb{C} \rightarrow \mathbb{R}$ such that:
$$ \forall z \in \mathbb{C}, f(z) = u(z) + iv(z)$$
Of course, $u$ and $v$ are the real and imaginary parts of $f(z)$.

\item and, notice a similar property about $z$:  $$\forall z \in \mathbb{C}, z = x + iy$$ where $x$ and $y$ are the imaginary parts of $z$.
\end{enumerate}

we can then study the following quantities:
\begin{itemize}
    \item $\frac{\partial f}{\partial x}$ and $\frac{\partial f}{\partial y}$
\item $\frac{\partial u}{\partial x}$, $\frac{\partial v}{\partial x}$, $\frac{\partial u}{\partial y}$ and $\frac{\partial v}{\partial y}$
\end{itemize}


\begin{thm*}[Cauchy-Riemann Equations]
$f$ is complex differentiable on $\Omega$ if and only if both:
\begin{enumerate}
    \item $f$ is real-differentiable (i.e. its differential exists, but maybe isn't $\mathbb{C}-linear$)
    \item its real-differential is $\mathbb{C}$-linear.
\end{enumerate}

The second condition (2) can be replaced by one of the following properties:

\begin{enumerate}
    \item[(a)] $ \forall z, df_z(i) = idf_z(1)$
    \item[(b)] $f$ verifies the \textbf{complex Cauchy-Riemann equation}:
        $$ \frac{\partial f}{\partial x} = \frac{1}{i}\frac{\partial f}{\partial y}$$
    \item[(b)] $f$ verifies the \textbf{scalar Cauchy-Riemann equations}:
        $$ \frac{\partial u}{\partial x} = +\frac{\partial v}{\partial y}$$
        $$ \frac{\partial u}{\partial y} = -\frac{\partial v}{\partial x}$$
\end{enumerate}
\end{thm*}

\begin{thm*}[Cauchy-Riemann Equations (alternate)]
$f$ is complex differentiable on $\Omega$ if and only if one of the following conditions hold:

\begin{enumerate}
    \item[(a)] 
        $ \frac{\partial f}{\partial x}$ and $ \frac{\partial f}{\partial y}$ exist, are \textbf{continuous} and verify the \textbf{complex Cauchy-Riemann equation}
        $$ \frac{\partial f}{\partial x} = \frac{1}{i}\frac{\partial f}{\partial y}$$
    \item[(b)] $\frac{\partial u}{\partial x}$, $\frac{\partial v}{\partial x}$, $\frac{\partial u}{\partial y}$ and $\frac{\partial v}{\partial y}$ exist, are \textbf{continous} and verify the \textbf{scalar Cauchy-Riemann equations}
        $$ \frac{\partial u}{\partial x} = +\frac{\partial v}{\partial y}$$
        $$ \frac{\partial u}{\partial y} = -\frac{\partial v}{\partial x}$$
\end{enumerate}
\end{thm*}
\begin{exo}[$\star$]
    Show that the complex exponential and logarithm maps are holomorphic where they are defined.    
\end{exo}
