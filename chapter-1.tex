\documentclass{article}
\title{Complex Analysis: lecture notes}
\author{Th\'eophile Cantelobre}
\usepackage{amsmath}
\usepackage{amsthm}
\usepackage{amsfonts}
\usepackage{hyperref}
\newtheorem*{defi}{Definition}
\newtheorem*{note}{Note}
\newtheorem{exo}{Exercise}
\newtheorem*{example}{Example}
\newtheorem*{schema}{Drawing}
\newtheorem*{thm*}{Theorem}

\begin{document}
\maketitle
\section{Introduction}
\subsection{C1123 Introduction}
Welcome to C1223! I hope you'll enjoy this very interesting and unique course at Mines ParisTech.

It is a great opportunity to learn about an important field in mathematics (including Applied Mathematics) and one of the last opportunities for you to stretch your ``fundamental" math skills before you become engineers.

Whether you've appreciated math courses in the past or not, there should be something in this course for you. The theory is both intuitive (we'll be drawing a lot) and beautiful (things are simple).
\subsubsection{Pre-requisites}
There are very few prerequisites for this course apart from basic linear algebra (vector spaces, ...), basic topology (open and closed sets, ...) and analysis (Taylor series, ...).

The textbook is mostly self-contained and I'll happily provide any background you may be lacking (or have forgotten) during the lectures.

\subsubsection{Methodological reminders}
The ``lectures" for C1223 last an hour and are directly followed by a problem sesssion (with another instructor). There are two such sessions per day for a week so the course is in fact quite intense.

Some advice:
\begin{description}
    \item[Stay focused!] The course is fast-paced and you won't have much time to absorb the material between sessions! However, rest assured that the material is tailor-made to fit to this format. Just follow the flow and the ideas will progressively mature over the week and following weeks before the exam. I'll try to provide context as we go. 
    \item [Ask questions!] The goal is for everyone to be able to make progress and asking questions definitely contributes to this. As soon as the first person asks his or hers, I'm confident the rest will follow!
    \item [Draw!] In Complex Analysis, a picture by proof is often more than half way to a solution. In fact, without drawings some parts of the material would be virtually impossible to present. 
    \item [Finally, give feedback!] This is a learning experience for me at least as much as it is for you. If something is unclear or poorly explained, ask questions! And if the pace is too slow or too fast, speak up! I'm happy to accomodate your requests.
\end{description}

\subsubsection{About this document}
Before we start, a few words about this document:

First, what is it not?
\begin{description}
    \item [A replacement textbook] S.B.'s text is extremely well written and we'll be using it as the main tool for the course. It contains the material, exercices and their solutions.
    \item [A cheat sheet] You should be getting one of those at the end of the course to help in your review before the exams.
\end{description}
What is it then?
\begin{description}
    \item [A summary] I try to present the most important elements of the text, without any proofs but with illustrations, examples, counter-examples and heuristics.
    \item [A study-guide] If you can master everything that is here, you should be fine.
    \item [My notes] I wrote this while preparing for TA'ing this course, so actually, it is mostly for me. I hope it will be useful for you too though.
\end{description}

A final note (and disclaimer): these pages are not a subset of the textbook material nor are they a superset... The syllabus will be explicited by the Professors responsible for the course. Nothing here is official nor endorsed by them. The material I've added are mostly reminders from prerequisites and extra counter-examples... basically, what I had scribbled in the margin of my textbook when I took the course.


\newpage
\section{First sesssion: Complex-differentiability} 
\subsection*{Contents}
\begin{enumerate}
    \item Reminders
        \begin{enumerate}
            \item topology: open sets (definition and examples)
            \item $\mathbb{C}$, $\mathbb{C}$-vector spaces, $\mathbb{C}$-linearity
            \item (Analysis in $\mathbb{R}$: characterisation of differentiability)
        \end{enumerate}
    \item Complex differentiability:
    \begin{enumerate}
        \item How do we define complex-differentiabiity? 
        \item the complex-derivative? 
        \item the complex-differential? 
        \item What is the difference between real-differentiability and complex-differentiability?
     \end{enumerate}
        
    \item Complex differentiability in practice
    \begin{enumerate}
        \item Usual theorems
        \item Cauchy-Reimann equations: two formulations
    \end{enumerate}
\end{enumerate}

\subsection{Quick reminders}

\subsubsection{Open sets}
\begin{defi}[Open set]
    A set $\Omega \subset A$ is open in $A$ when:$ \forall a\in\Omega, \exists r > 0, \forall y \in \mathcal{B}_r(a), y \in \Omega $
\end{defi}

Alternatively:
    \begin{itemize}
        \item $ \forall a\in\Omega, \exists r > 0, \forall y \in A, ||y - a|| < r \implies y \in \Omega $
        \item $\mathring \Omega = \Omega$
    \end{itemize}
In practice:
\begin{enumerate}
    \item It can be "clear": the goal of this course is not to show that sets are open. 
    \item Write $\Omega$ as the inverse image of an open set by a continuous map. For example, the open unit ball is the inverse image of $]0,1[$ by $x\in E \mapsto ||x||$.
\end{enumerate}
\begin{schema} Visually, this means you can get as close as you want to the complement of the open set $\Omega$ and your neighbors are still in $\Omega$.
\end{schema}

\begin{example}
A few examples of open sets are: $\mathbb{R}$, $\mathbb{C}$, $]0,1[$. $[0,1]$ and $[0,1[$ are not open.
\end{example}

\subsubsection{The complex plane \& complex functions}
Definitions and properties:
\begin{enumerate}
    \item Definition: $\lbrace x+iy, x\in\mathbb{R}, y\in\mathbb{R} \rbrace$, $i$ such that $i^2=1$.
    \item Exponential, trigonometric forms: $$\exists R>0, \theta, x+iy = R(\cos\theta + i\sin\theta)=R\exp(i\theta) $$
    \item Conjugate: $z=x+iy \mapsto \bar z = x - iy$, $z = Re^{i\theta} \mapsto \bar z = Re^{-i\theta}$
    \item \dots 
\end{enumerate}

\begin{exo}
    On the structure of $\mathbb{C}$ as a vector space.
    \begin{enumerate}
        \item Show that the complex plane $\mathbb{C}$ is a complex vector space. 
        \item Give a basis. 
        \item What changes if we consider it to be a real vector space?
    \end{enumerate}
\end{exo}

\begin{exo}
    Give an example of a $\mathbb{C}$-linear function. Give a counter-example (an $\mathbb{R}$-linear function from $\mathbb{C}$ to $\mathbb{C}$ that is not $\mathbb{C}$-linear.)
\end{exo}


\subsection{Complex-differentiable, derivative and differential}
In this section, $\Omega$ is an open subset of $\mathbb{C}$.

\begin{defi}[Complex-differentiability \& Derivative]
    Let $z$ an interior point in $\Omega$. Then, $f$ is \emph{complex-differentiable} at $z$ iff the \emph{complex derivative} of $f$ exists in $\mathbb{C}$, as defined by 
    $$ f'(z) = \lim_{h\rightarrow 0, h\in\mathbb{C}}\frac{f(z+h) - f(z)}{h} $$ 
\end{defi}

\begin{note}
    Let's insist on the $h\in\mathbb{C}$ in the definition. This means that the limit exists no matter the direction from which we are "arriving" in $\mathbb{C}$. Also, no matter the direction, the value is the same.
\end{note}
\begin{note}
    The previous definition is a \emph{local} property. Let's generalize it to $\Omega$ below.
\end{note}

\begin{defi}[Holomorphic]
    $f: \mathbb{C} \rightarrow \mathbb{C}$ is \emph{holomorphic} if $\forall z\in\Omega$, it is \emph{complex-differentiable}, i.e. \emph{complex-differentiable everywhere.}
\end{defi}
 
\begin{example} Some examples of holomorphic functions. Also, a counter-example.
    \begin{itemize}
        \item Some "obvious" examples: constant functions, identity, affine, \dots
        \item ($\star$) Show the inverse function is holomorphic where it is defined. 
        \item ($\star$) A counter example: the complex conjuguate function is nowhere complex-differentiable.
    \end{itemize}
\end{example}

\begin{defi}
    Let $f: A\subset \mathbb{C} \rightarrow\mathbb{C} $.

    Let $z\in A$ as interior point. 

    We define the \emph{complex-differential} (or $\mathbb{C}$-differential) of $f$ in $z$ a complex-linear, continous operator $df_x: \mathbb{C} \rightarrow \mathbb{C} $ that verifies:
        $$ \lim_{h\rightarrow 0, h\in\mathbb{C}}\frac{||f(z+h) - f(z) - df_z(h)||}{||h||} = 0 $$ 
    Equivalently:
        $$ f(z+h) = f(z) + df_z(h) + o(h) $$ 
\end{defi}
\begin{note}
    Recall that $$g(h) = o(h) \iff \lim_{h\rightarrow 0}\frac{g(h)}{h} =0 $$
\end{note}
\begin{note}
    This result extends to $f : E \rightarrow F$ where $E$ and $F$ are complex, normed vector spaces. For example, $\mathbb{C}^n$.
\end{note}


Now, an important theorem that reflects the structure of $\mathbb{C}$:
\begin{thm*}
    Let $f: A \subset\mathbb{C} \rightarrow \mathbb{C}$. Let $z \in \mathring A$. 
    
    The complex-differential of $f$ $df_z$ exists iff the derivative $f'(z)$ exists.
    
    In this case, we have:
    $$\forall h \in \mathbb{C}, df_z(h) = f'(z)h$$
\end{thm*}

\begin{note}
    This is linked to our view of $\mathbb{C}$ as a complex vector space, of dimension $1$.
\end{note}

\subsection{Complex-differentiability in practice}

\subsubsection{Calculus}
\begin{itemize}
    \item $\mathbb{C}$-linear combination
    \item Product
    \item Chain rule
    \item Quotient
    \item Polynomials
    \item Rational fractions
\end{itemize}

In practice: \emph{calculate like in $\mathbb{R}$}.

These "usual" properties allow us to verify that functions are $\mathbb{C}$-differentiable or holomorphic easily, when they are products and compositions (for example).

\subsubsection{Cauchy-Reimann Equations}

We can rephrase the definitions as:
\begin{thm*}
$f$ is complex differentiable on $\Omega$ if and only if both:
\begin{enumerate}
    \item $f$ is real-differentiable on $\Omega$ (i.e. its differential exists, but maybe isn't $\mathbb{C}-linear$)
    \item its real-differential is $\mathbb{C}$-linear.
\end{enumerate}
\end{thm*}

But the second condition is not easy to prove without expliciting the differential. The Cauchy-Riemann equations offer an alternative formulation of this condition.

The intuition behind the theorem is to look at what happens to the real and imaginary parts of $f$ when we move the real and imaginary parts of $z$

If we write:
\begin{enumerate}
    \item if $f: \mathbb{C} \rightarrow \mathbb{C}$, we have $u: \mathbb{C} \rightarrow \mathbb{R}$ and $v: \mathbb{C} \rightarrow \mathbb{R}$ such that:
$$ \forall z \in \mathbb{C}, f(z) = u(z) + iv(z)$$
Of course, $u$ and $v$ are the real and imaginary parts of $f(z)$.

\item and, notice a similar property about $z$:  $$\forall z \in \mathbb{C}, z = x + iy$$ where $x$ and $y$ are the imaginary parts of $z$.
\end{enumerate}

we can then study the following quantities:
\begin{itemize}
    \item $\frac{\partial f}{\partial x}$ and $\frac{\partial f}{\partial y}$
\item $\frac{\partial u}{\partial x}$, $\frac{\partial v}{\partial x}$, $\frac{\partial u}{\partial y}$ and $\frac{\partial v}{\partial y}$
\end{itemize}


\begin{thm*}[Cauchy-Riemann Equations]
$f$ is complex differentiable on $\Omega$ if and only if both:
\begin{enumerate}
    \item $f$ is real-differentiable (i.e. its differential exists, but maybe isn't $\mathbb{C}-linear$)
    \item its real-differential is $\mathbb{C}$-linear.
\end{enumerate}

The second condition (2) can be replaced by one of the following properties:

\begin{enumerate}
    \item[(a)] $ \forall z, df_z(i) = idf_z(1)$
    \item[(b)] $f$ verifies the \textbf{complex Cauchy-Riemann equation}:
        $$ \frac{\partial f}{\partial x} = \frac{1}{i}\frac{\partial f}{\partial y}$$
    \item[(b)] $f$ verifies the \textbf{scalar Cauchy-Riemann equations}:
        $$ \frac{\partial u}{\partial x} = +\frac{\partial v}{\partial y}$$
        $$ \frac{\partial u}{\partial y} = -\frac{\partial v}{\partial x}$$
\end{enumerate}
\end{thm*}

\begin{thm*}[Cauchy-Riemann Equations (alternate)]
$f$ is complex differentiable on $\Omega$ if and only if one of the following conditions hold:

\begin{enumerate}
    \item[(a)] 
        $ \frac{\partial f}{\partial x}$ and $ \frac{\partial f}{\partial y}$ exist, are \textbf{continuous} and verify the \textbf{complex Cauchy-Riemann equation}
        $$ \frac{\partial f}{\partial x} = \frac{1}{i}\frac{\partial f}{\partial y}$$
    \item[(b)] $\frac{\partial u}{\partial x}$, $\frac{\partial v}{\partial x}$, $\frac{\partial u}{\partial y}$ and $\frac{\partial v}{\partial y}$ exist, are \textbf{continous} and verify the \textbf{scalar Cauchy-Riemann equations}
        $$ \frac{\partial u}{\partial x} = +\frac{\partial v}{\partial y}$$
        $$ \frac{\partial u}{\partial y} = -\frac{\partial v}{\partial x}$$
\end{enumerate}
\end{thm*}
\begin{exo}[$\star$]
    Show that the complex exponential and logarithm maps are holomorphic where they are defined.    
\end{exo}

\end{document}
