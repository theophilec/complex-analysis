\documentclass{article}
\title{Complex Analysis: lecture notes}
\author{Th\'eophile Cantelobre}
\usepackage[margin=1.0in]{geometry}
\usepackage{amsmath}
\usepackage{amsthm}
\usepackage{amssymb}
\usepackage{amsfonts}
\usepackage{hyperref}
\newtheorem*{defi}{Definition}
\newtheorem*{note}{Note}
\newtheorem{exo}{Exercise}
\newtheorem*{example}{Example}
\newtheorem*{schema}{Drawing}
\newtheorem*{thm*}{Theorem}

% redefine \emptyset as \varnothing
\let\oldemptyset\emptyset
\let\emptyset\varnothing


\begin{document}
\maketitle
\tableofcontents

\newpage
\section*{Introduction}
\subsection{C1123 Introduction}
Welcome to C1223! I hope you'll enjoy this very interesting and unique course at Mines ParisTech.

It is a great opportunity to learn about an important field in mathematics (including Applied Mathematics) and one of the last opportunities for you to stretch your ``fundamental" math skills before you become engineers.

Whether you've appreciated math courses in the past or not, there should be something in this course for you. The theory is both intuitive (we'll be drawing a lot) and beautiful (things are simple).
\subsubsection{Pre-requisites}
There are very few prerequisites for this course apart from basic linear algebra (vector spaces, ...), basic topology (open and closed sets, ...) and analysis (Taylor series, ...).

The textbook is mostly self-contained and I'll happily provide any background you may be lacking (or have forgotten) during the lectures.

\subsubsection{Methodological reminders}
The ``lectures" for C1223 last an hour and are directly followed by a problem sesssion (with another instructor). There are two such sessions per day for a week so the course is in fact quite intense.

Some advice:
\begin{description}
    \item[Stay focused!] The course is fast-paced and you won't have much time to absorb the material between sessions! However, rest assured that the material is tailor-made to fit to this format. Just follow the flow and the ideas will progressively mature over the week and following weeks before the exam. I'll try to provide context as we go. 
    \item [Ask questions!] The goal is for everyone to be able to make progress and asking questions definitely contributes to this. As soon as the first person asks his or hers, I'm confident the rest will follow!
    \item [Draw!] In Complex Analysis, a picture by proof is often more than half way to a solution. In fact, without drawings some parts of the material would be virtually impossible to present. 
    \item [Finally, give feedback!] This is a learning experience for me at least as much as it is for you. If something is unclear or poorly explained, ask questions! And if the pace is too slow or too fast, speak up! I'm happy to accomodate your requests.
\end{description}

\subsubsection{About this document}
Before we start, a few words about this document:

First, what is it not?
\begin{description}
    \item [A replacement textbook] S.B.'s text is extremely well written and we'll be using it as the main tool for the course. It contains the material, exercices and their solutions.
    \item [A cheat sheet] You should be getting one of those at the end of the course to help in your review before the exams.
\end{description}
What is it then?
\begin{description}
    \item [A summary] I try to present the most important elements of the text, without any proofs but with illustrations, examples, counter-examples and heuristics.
    \item [A study-guide] If you can master everything that is here, you should be fine.
    \item [My notes] I wrote this while preparing for TA'ing this course, so actually, it is mostly for me. I hope it will be useful for you too though.
\end{description}

A final note (and disclaimer): these pages are not a subset of the textbook material nor are they a superset... The syllabus will be explicited by the Professors responsible for the course. Nothing here is official nor endorsed by them. The material I've added are mostly reminders from prerequisites and extra counter-examples... basically, what I had scribbled in the margin of my textbook when I took the course.


\newpage
\section{First sesssion: Complex-differentiability} 
\subsection*{Contents}
\begin{enumerate}
    \item Reminders
        \begin{enumerate}
            \item topology: open sets (definition and examples)
            \item $\mathbb{C}$, $\mathbb{C}$-vector spaces, $\mathbb{C}$-linearity
            \item (Analysis in $\mathbb{R}$: characterisation of differentiability)
        \end{enumerate}
    \item Complex differentiability:
    \begin{enumerate}
        \item How do we define complex-differentiabiity? 
        \item the complex-derivative? 
        \item the complex-differential? 
        \item What is the difference between real-differentiability and complex-differentiability?
     \end{enumerate}
        
    \item Complex differentiability in practice
    \begin{enumerate}
        \item Usual theorems
        \item Cauchy-Reimann equations: two formulations
    \end{enumerate}
\end{enumerate}

\subsection{Quick reminders}

\subsubsection{Open sets}
\begin{defi}[Open set]
    A set $\Omega \subset A$ is open in $A$ when:$ \forall a\in\Omega, \exists r > 0, \forall y \in \mathcal{B}_r(a), y \in \Omega $
\end{defi}

Alternatively:
    \begin{itemize}
        \item $ \forall a\in\Omega, \exists r > 0, \forall y \in A, ||y - a|| < r \implies y \in \Omega $
        \item $\mathring \Omega = \Omega$
    \end{itemize}
In practice:
\begin{enumerate}
    \item It can be "clear": the goal of this course is not to show that sets are open. 
    \item Write $\Omega$ as the inverse image of an open set by a continuous map. For example, the open unit ball is the inverse image of $]0,1[$ by $x\in E \mapsto ||x||$.
\end{enumerate}
\begin{schema} Visually, this means you can get as close as you want to the complement of the open set $\Omega$ and your neighbors are still in $\Omega$.
\end{schema}

\begin{example}
A few examples of open sets are: $\mathbb{R}$, $\mathbb{C}$, $]0,1[$. $[0,1]$ and $[0,1[$ are not open.
\end{example}

\subsubsection{The complex plane \& complex functions}
Definitions and properties:
\begin{enumerate}
    \item Definition: $\lbrace x+iy, x\in\mathbb{R}, y\in\mathbb{R} \rbrace$, $i$ such that $i^2=1$.
    \item Exponential, trigonometric forms: $$\exists R>0, \theta, x+iy = R(\cos\theta + i\sin\theta)=R\exp(i\theta) $$
    \item Conjugate: $z=x+iy \mapsto \bar z = x - iy$, $z = Re^{i\theta} \mapsto \bar z = Re^{-i\theta}$
    \item \dots 
\end{enumerate}

\begin{exo}
    On the structure of $\mathbb{C}$ as a vector space.
    \begin{enumerate}
        \item Show that the complex plane $\mathbb{C}$ is a complex vector space. 
        \item Give a basis. 
        \item What changes if we consider it to be a real vector space?
    \end{enumerate}
\end{exo}

\begin{exo}
    Give an example of a $\mathbb{C}$-linear function. Give a counter-example (an $\mathbb{R}$-linear function from $\mathbb{C}$ to $\mathbb{C}$ that is not $\mathbb{C}$-linear.)
\end{exo}


\subsection{Complex-differentiable, derivative and differential}
In this section, $\Omega$ is an open subset of $\mathbb{C}$.

\begin{defi}[Complex-differentiability \& Derivative]
    Let $z$ an interior point in $\Omega$. Then, $f$ is \emph{complex-differentiable} at $z$ iff the \emph{complex derivative} of $f$ exists in $\mathbb{C}$, as defined by 
    $$ f'(z) = \lim_{h\rightarrow 0, h\in\mathbb{C}}\frac{f(z+h) - f(z)}{h} $$ 
\end{defi}

\begin{note}
    Let's insist on the $h\in\mathbb{C}$ in the definition. This means that the limit exists no matter the direction from which we are "arriving" in $\mathbb{C}$. Also, no matter the direction, the value is the same.
\end{note}
\begin{note}
    The previous definition is a \emph{local} property. Let's generalize it to $\Omega$ below.
\end{note}

\begin{defi}[Holomorphic]
    $f: \mathbb{C} \rightarrow \mathbb{C}$ is \emph{holomorphic} if $\forall z\in\Omega$, it is \emph{complex-differentiable}, i.e. \emph{complex-differentiable everywhere.}
\end{defi}
 
\begin{example} Some examples of holomorphic functions. Also, a counter-example.
    \begin{itemize}
        \item Some "obvious" examples: constant functions, identity, affine, \dots
        \item ($\star$) Show the inverse function is holomorphic where it is defined. 
        \item ($\star$) A counter example: the complex conjuguate function is nowhere complex-differentiable.
    \end{itemize}
\end{example}

\begin{defi}
    Let $f: A\subset \mathbb{C} \rightarrow\mathbb{C} $.

    Let $z\in A$ as interior point. 

    We define the \emph{complex-differential} (or $\mathbb{C}$-differential) of $f$ in $z$ a complex-linear, continous operator $df_x: \mathbb{C} \rightarrow \mathbb{C} $ that verifies:
        $$ \lim_{h\rightarrow 0, h\in\mathbb{C}}\frac{||f(z+h) - f(z) - df_z(h)||}{||h||} = 0 $$ 
    Equivalently:
        $$ f(z+h) = f(z) + df_z(h) + o(h) $$ 
\end{defi}
\begin{note}
    Recall that $$g(h) = o(h) \iff \lim_{h\rightarrow 0}\frac{g(h)}{h} =0 $$
\end{note}
\begin{note}
    This result extends to $f : E \rightarrow F$ where $E$ and $F$ are complex, normed vector spaces. For example, $\mathbb{C}^n$.
\end{note}


Now, an important theorem that reflects the structure of $\mathbb{C}$:
\begin{thm*}
    Let $f: A \subset\mathbb{C} \rightarrow \mathbb{C}$. Let $z \in \mathring A$. 
    
    The complex-differential of $f$ $df_z$ exists iff the derivative $f'(z)$ exists.
    
    In this case, we have:
    $$\forall h \in \mathbb{C}, df_z(h) = f'(z)h$$
\end{thm*}

\begin{note}
    This is linked to our view of $\mathbb{C}$ as a complex vector space, of dimension $1$.
\end{note}

\subsection{Complex-differentiability in practice}

\subsubsection{Calculus}
\begin{itemize}
    \item $\mathbb{C}$-linear combination
    \item Product
    \item Chain rule
    \item Quotient
    \item Polynomials
    \item Rational fractions
\end{itemize}

In practice: \emph{calculate like in $\mathbb{R}$}.

These "usual" properties allow us to verify that functions are $\mathbb{C}$-differentiable or holomorphic easily, when they are products and compositions (for example).

\subsubsection{Cauchy-Reimann Equations}

We can rephrase the definitions as:
\begin{thm*}
$f$ is complex differentiable on $\Omega$ if and only if both:
\begin{enumerate}
    \item $f$ is real-differentiable on $\Omega$ (i.e. its differential exists, but maybe isn't $\mathbb{C}-linear$)
    \item its real-differential is $\mathbb{C}$-linear.
\end{enumerate}
\end{thm*}

But the second condition is not easy to prove without expliciting the differential. The Cauchy-Riemann equations offer an alternative formulation of this condition.

The intuition behind the theorem is to look at what happens to the real and imaginary parts of $f$ when we move the real and imaginary parts of $z$

If we write:
\begin{enumerate}
    \item if $f: \mathbb{C} \rightarrow \mathbb{C}$, we have $u: \mathbb{C} \rightarrow \mathbb{R}$ and $v: \mathbb{C} \rightarrow \mathbb{R}$ such that:
$$ \forall z \in \mathbb{C}, f(z) = u(z) + iv(z)$$
Of course, $u$ and $v$ are the real and imaginary parts of $f(z)$.

\item and, notice a similar property about $z$:  $$\forall z \in \mathbb{C}, z = x + iy$$ where $x$ and $y$ are the imaginary parts of $z$.
\end{enumerate}

we can then study the following quantities:
\begin{itemize}
    \item $\frac{\partial f}{\partial x}$ and $\frac{\partial f}{\partial y}$
\item $\frac{\partial u}{\partial x}$, $\frac{\partial v}{\partial x}$, $\frac{\partial u}{\partial y}$ and $\frac{\partial v}{\partial y}$
\end{itemize}


\begin{thm*}[Cauchy-Riemann Equations]
$f$ is complex differentiable on $\Omega$ if and only if both:
\begin{enumerate}
    \item $f$ is real-differentiable (i.e. its differential exists, but maybe isn't $\mathbb{C}-linear$)
    \item its real-differential is $\mathbb{C}$-linear.
\end{enumerate}

The second condition (2) can be replaced by one of the following properties:

\begin{enumerate}
    \item[(a)] $ \forall z, df_z(i) = idf_z(1)$
    \item[(b)] $f$ verifies the \textbf{complex Cauchy-Riemann equation}:
        $$ \frac{\partial f}{\partial x} = \frac{1}{i}\frac{\partial f}{\partial y}$$
    \item[(b)] $f$ verifies the \textbf{scalar Cauchy-Riemann equations}:
        $$ \frac{\partial u}{\partial x} = +\frac{\partial v}{\partial y}$$
        $$ \frac{\partial u}{\partial y} = -\frac{\partial v}{\partial x}$$
\end{enumerate}
\end{thm*}

\begin{thm*}[Cauchy-Riemann Equations (alternate)]
$f$ is complex differentiable on $\Omega$ if and only if one of the following conditions hold:

\begin{enumerate}
    \item[(a)] 
        $ \frac{\partial f}{\partial x}$ and $ \frac{\partial f}{\partial y}$ exist, are \textbf{continuous} and verify the \textbf{complex Cauchy-Riemann equation}
        $$ \frac{\partial f}{\partial x} = \frac{1}{i}\frac{\partial f}{\partial y}$$
    \item[(b)] $\frac{\partial u}{\partial x}$, $\frac{\partial v}{\partial x}$, $\frac{\partial u}{\partial y}$ and $\frac{\partial v}{\partial y}$ exist, are \textbf{continous} and verify the \textbf{scalar Cauchy-Riemann equations}
        $$ \frac{\partial u}{\partial x} = +\frac{\partial v}{\partial y}$$
        $$ \frac{\partial u}{\partial y} = -\frac{\partial v}{\partial x}$$
\end{enumerate}
\end{thm*}
\begin{exo}[$\star$]
    Show that the complex exponential and logarithm maps are holomorphic where they are defined.    
\end{exo}

\newpage
\section{Second sesssion: Line Integrals \& Primitives} 
The goal of this chapter is to derive an equivalent to the fundamental theorem of calculus for functions of a complex variable.

Recall that is $\mathbb{R}$ one generally uses a rectangular approximation along the segment one integrates over. In $\mathbb{C}$, the principle is the same: we'll present how we can build paths on $\mathbb{C}$ and then how we can integrate over them by approximating the paths.
\subsection*{Contents}
\begin{enumerate}
    \item Paths
        \begin{enumerate}
            \item Definition \& Examples
            \item Path concatenation
            \item Rectifiable paths
            \item Connected sets
        \end{enumerate}
    \item Line integrals 
        \begin{enumerate}
            \item Definition
            \item Calculus
            \item Reparametrization \& Change of variables
            \item M-L Inequality
        \end{enumerate}
    \item Primitives
        \begin{enumerate}
            \item Definition \& Fundamental theorem of calculus (complex analysis)
            \item Useful theorems, integration by parts
        \end{enumerate}
        
\end{enumerate}

\subsection{Paths}
\subsubsection{Definition, Vocabulary \& Examples}
\begin{defi}[Path]
    A path is a continuous function from $[0, 1]$ to $\mathbb{C}$.
\end{defi}

\begin{note}
    The intuition behind a path is that it is parametrized by time, or another variable independent of the complex plane.
\end{note}

\begin{exo}[Unit circle]
    Give a path whose image is the unit circle.
\end{exo}

\begin{exo}[Straight line]
    Give a path such that $\gamma(0) = a$ and $\gamma(1) = b$ where $a, b \in \mathbb{C}$. 
\end{exo}

\begin{note}
    Paths are oriented!
\end{note}

\begin{exo}[Reverse path]
    Given a path $\gamma$ how would you describe its reverse path $\gamma^{\leftarrow}$.
\end{exo}


\subsubsection{Image of a path}
\begin{defi}[Image of a path]
    The image (or trajectory, or trace) of a path $\gamma$ is $\gamma([0,1])$, all the points reached by the path.
\end{defi}

\begin{defi}[Path on a subset]
    A path is \emph{on $A\subset\mathbb{C}$} if its image is a subset of $A$, i.e. $\gamma([0,1])\subset A$.
\end{defi}

\begin{exo}
    The image of a path is compact.
\end{exo}

\begin{proof}[Solution]
    $\gamma$ is continuous and $[0,1]$ is compact (closed and bounded, in a finite dimensional space), so $\gamma([0,1])$ is compact.
\end{proof}

\subsubsection{Path concatenation}

Here we present the vocabulary and properties necessary to build more complex paths.
\begin{defi}[Consecutive paths]
    Two paths $\gamma_1$ and $\gamma_2$ are consecutive if $\gamma_1(1) = \gamma_2(0)$, i.e. if $\gamma_1$ terminates (terminal point) where $\gamma_2$ begins (initial point).
\end{defi}

\begin{defi}[Path concatenation]
    Let $t_0=0 < t_1 < \dots < t_{n-1} < 1=t_n $ be a partition of $[0,1]$ (these are our endpoints).

    Let $\gamma_1$, \dots, $\gamma_n$ $n$ consecutive paths (these are the paths.)

    The concatenation of $\gamma_1, \dots, \gamma_n$ associated to the partition $t_0, \dots, t_{n-1}$ is the path $\gamma$ uniquely defined such that:
    $$ \boxed{\forall k \in \lbrace0, \dots, n-1\rbrace, \gamma_{|[t_{k-1}, t_{k}]}= \gamma_k\left( \frac{t-t_{k-1}}{t_k - t_{k-1}} \right)}$$
    We denote it:
    $$ \gamma_1 |_{t_1} \dots |_{t_{n-1}} \gamma_n$$
\end{defi}

We can simplify notations if the partition is uniform (i.e. $t_k = k/n$):
    $$ \gamma_1 | \dots | \gamma_n$$

\begin{example}[Oriented Polyline]
    An oriented polyline is the concatenation of consecutive oriented line segments. We denote them:
    $$[a_0 \rightarrow \dots \rightarrow a_n]$$
    where $(a_0, \dots, a_n)\in\mathbb{C}^n$ are "the endpoints".
\end{example}

\subsubsection{Paths \& Regularity}

Until now, the paths we've described were "only" continuous. Integrating over paths is like using changes of variable. Recall that is real analysis, changes of variable generally need to be $\mathcal{C}^1$. Here we define analogous tools in $\mathbb{C}$.

\begin{defi}
    A path is rectifiable if it is piece-wise continuously differentiable, i.e. it is differentiable and its differential is continuous.
\end{defi}

\begin{note}[Continuous differential]
    Let's clarify the meaning of continuous in this case. For this, recall that $f$'s differential at a point $z$ is a linear operator $df_z$. This operator is continuous with respect to its argument, i.e. $h \mapsto df_z(h)$ is continuous.

    A function $f$ is continously differentiable if the map $z \mapsto df_z$ is continuous.

    In the case of paths, this comes down to the path being differentiable and its derivative being continuous.
\end{note}

\begin{note}
    There are no conditions on the differentials at the boudaries "between pieces".
\end{note}

\begin{example}
    An oriented polyline is rectifiable.
\end{example}

\begin{thm*}[Continuously differentiable decomposition]
    A path $\gamma$ is rectifiable if and only if there are $\gamma_1, \dots, \gamma_n$ consecutive continuously differentiable paths and a partition $(t_1, \dots, t_{n-1})$ of $[0,1]$ such that:
    $$\gamma = \gamma_1 |_{t_1} \dots |_{t_{n-1}} \gamma_n$$
\end{thm*}

\subsubsection{Connected sets}

\begin{defi}
    An open subset $\Omega$ of $\mathbb{C}$ is (path-)connected if for any points $x,y \in \Omega^2$, there is a path on $\Omega$ that joins $x$ and $y$. 
\end{defi}

\begin{example}
    
    Some connected open sets:
    \begin{itemize}
        \item $\mathbb{C}$
        \item $\mathcal{B}_0(r)$
        \item \dots
    \end{itemize}
    Some open, disconnected sets:
    \begin{itemize}
        \item $\mathbb{C} -\mathbb{R}i$
        \item $\mathcal{B}_0(r) - \lbrace 0 \rbrace$
        \item \dots
    \end{itemize}
\end{example}

\begin{thm*}
    An open subset $\Omega$ of the complex plane is connected if and only if every pair of points can be joined by a rectifiable path of $\Omega$.
\end{thm*}

\begin{proof}[Sketch of the proof]
    $\gamma([0,1])$ is compact and $\mathbb{C} - \Omega$ is closed. Thus the difference between the two is strictly positive. By uniform continuity of $\gamma$ we can build an oriented polyline that approximates it while staying within $\Omega$ (same technique as is $\mathbb{R}$). The polyline is a rectifiable path on $\Omega$.
\end{proof}

\subsection{Line Integrals}
In this section, we define the line integral of $f:\mathbb{C} \rightarrow \mathbb{C}$ over a rectifiable path $\gamma$ as the integral over $[0,1]$ of $f\circ\gamma \times \gamma'$.

\begin{defi}
    The line integral along a rectifiable path $\gamma$ of $f:\mathbb{C} \rightarrow \mathbb{C}$, continuous over $\gamma([0,1])$ is:

    $$ \int_\gamma f(z)dz = \int_0^1 f(\gamma(t))\gamma'(t)dt$$
\end{defi}

\begin{note}
    We deliberately overlook the fact that $\gamma$ is only rectifiable and thus $\gamma'$ is undefined at a finite number of points (almost everywhere). 
\end{note}

This gives us a way of explicitly calculating line integrals! We won't be doing (too) much of that here. However, let's look at a few examples.

\begin{defi}[Length of a rectifiable path]
    The length of a rectifiable path $\gamma$ is 
    $$ \ell(\gamma) = \int_0^1 |\gamma'(t)|dt$$
\end{defi}

\begin{note}
    Because of the derivative, the "position" of $\gamma$ in the complex plane has no impact on the length of the path... this is what we'd expect!
\end{note}

\begin{exo}[$\star$]
    Calculate the length of a line segment. 
\end{exo}
\begin{exo}[$\star$]
    Show that $\ell(\gamma) = \ell(\gamma^{\leftarrow})$.
    If $\gamma = \gamma_1 |_{t_1} \dots |_{t_{n-1}} \gamma_n$ a concatenaiton of consecutive, rectifiable paths, show that:
    $$ \ell(\gamma) = \sum_k{\ell(\gamma_k)}$$
\end{exo}
\begin{exo}[$\star$]
    If $\gamma = \gamma_1 |_{t_1} \dots |_{t_{n-1}} \gamma_n$ a concatenaiton of consecutive, rectifiable paths, show that:
    $$ \ell(\gamma) = \sum_k{\ell(\gamma_k)}$$
\end{exo}
\begin{exo}[$\star$]
    Calculate the length of an oriented circle of radius $r$ and with $n$ traversals.
\end{exo}
\begin{exo}
    Give a parametrization for the line integral of $f$ over $[a \rightarrow b]$.
\end{exo}
\begin{exo}
    Give a parametrization for the line integral of $f$ over the oriented circle of radius $r$, of center $0$ and in the positive direction. And with $n$ traversals?
\end{exo}


\subsubsection{Line integral calculus}
\begin{description}
    \item[Complex-linearity] $$\int_\gamma \alpha f + \beta g dz = \alpha \int_\gamma fdz + \beta \int_\gamma gdz$$
    \item[Integration along a reverse path] $$\int_\gamma fdz = - \int_{\gamma^{\leftarrow}}fdz$$
    \item[Integration over a concatenation] If $\gamma = \gamma_1 |_{t_1} \dots |_{t_{n-1}} \gamma_n$ a concatenaiton of consecutive, rectifiable paths, $\gamma$ is rectifiable and:
        $$\int_\gamma fdz = \sum_k \int_{\gamma_k}fdz$$
\end{description}

\subsubsection{Reparametrization \& Changes in variables in line integrals}
Here, we present two useful properties of line integrals: first, that we can reparametrize the path along which we are integrating (i.e. scale time) without changing the value of the integral; second, that we can apply classical change in variable formulas under certain conditions.

\begin{thm*}[Invariance by reparametrization]
Let $f: \mathbb{C} \rightarrow \mathbb{C}$ is a continuous function, and $\gamma$ a continuously differentiable path.

Let $\phi: [0,1] \rightarrow [0,1]$ an increasing-$\mathcal{C}^1$-diffeomorphism, i.e such that:
\begin{itemize}
    \item $\phi$ is continuously differentiable
    \item $\phi$ is increasing (i.e. $\phi'(t) > 0$)
    \item $\phi(0) = 0$ and $\phi(1)=1$.
\end{itemize}

Then, if $\mu = \gamma \circ \phi$:
\begin{enumerate}
    \item $\mu$ is a continuously differentiable path with the same endpoints and image as $\gamma$.
    \item $\ell(\mu) = \ell(\gamma)$
    \item The line integrals of $f$ over $\mu$ and $\gamma$ are equal:
        $$ \int_\mu fdz = \int_\gamma fdz $$
\end{enumerate}
\end{thm*}

\begin{note}
    The intutition behind this is that a path is a trajectory on the complex plane, for example, that of a robot. If two robots follow the same path at different speeds but leave and arrive simulataneously, their distances traveled are the same, and they've covered the same ground.
\end{note}

Now let's show that the usual change of variable formula holds in complex analysis:

\begin{thm*}[Changes of variables in line integrals]
    Let $\Omega$ be an open subset of $\mathbb{C}$.

    Let $f: \Omega \rightarrow \mathbb{C}$ be a holomorphic function.

    Let $g: f(\gamma([0,1])) \rightarrow \mathbb{C}$ a continuous function.

    Then,
    \begin{enumerate}
        \item $f\circ gamma$ is a rectifiable path.
        \item The following change of variables holds:
            $$\int_{f \circ \gamma}g(z)dz = \int_\gamma g\circ f(z)f'(z)dz$$
    \end{enumerate}

\end{thm*}

\subsubsection{ML-Inequality \& Convergence}

An important consequence of the triangular inequality is that the (line) integral can be bounded by the product of the maximum of the integrand and the length of the path (or segment) one integrates over.

In complex analysis, we have the following important result:

\begin{thm*}[M-L Inequality]
    Let $\gamma$ be a rectifiable path.

    Let $f: A\subset\mathbb{C} \rightarrow \mathbb{C}$ a continuous function.

    Then, 
    $$\boxed{\left\| \int_\gamma f \right\| \leq \max_{z\in \gamma([0,1])}{\| f(z)\|} \times \ell(\gamma)}$$
\end{thm*}

\begin{proof}
    Use the triangular inequality.
\end{proof}

Thanks to this inequality, we can prove that the line integrals of an uniform approximation $f_n$ of $f$ converge towards the the line integral of $f$, as in $\mathbb{R}$.

\begin{thm*}
    For any rectifiable path $\gamma$ and uniform approximation $f_n$ of $f$ a continuous function (i.e. $\lim_{n\rightarrow \infty}\left\| f_n -f \right \|_{\infty} = 0$), 
    $$ \lim_{n\rightarrow\infty} \int_\gamma f_n(z)dz = \int_\gamma f(z)dz $$
\end{thm*}

\subsection{Primitives}

Now that we have defined line integrals, we can identify the main difficulty of defining a primitive of a complex function: the integral depends on the the path use to integrate over! However, derivation is a very local property that does not know what happens far from the derivation point. For these reasons, the fundamental theorem of analysis over $\mathbb{R}$ is ambiguous. Here we adapt it, which yields important results about holomorphic functions.

\begin{defi}
    A primitive of a continuous function $f$ defined on a open subset $\Omega$ of $\mathbb{C}$ is a holomorphic function defined on $\Omega$ such that $g'=f$.
\end{defi}

This formulation is only useful as a definition (or in very simple concrete cases). The following characterisation of primitives is important:

\begin{thm*}[Fundamental theorem of calculus for complex analysis]
    Let $f: \Omega \rightarrow \mathbb{C}$ be a continuous function, $\Omega\subset\mathbb{C}$ open, connected.

    A function $g: \Omega \rightarrow \mathbb{C}$ is a primitive of $f$ if and only if: for any $z\in\Omega$ and any rectifiable path $\gamma$ on $\Omega$ that joins $a$ and $z$, 
    $$ g(z) = g(a) + \int_\gamma f(w)dw $$
\end{thm*}

\begin{note}
    Notice that the integral characterisation must be true for all paths. The $\mathbb{R}$ equivalent is weaker because there is unique path along which to integrate (modulo reparametrization).
\end{note}

\begin{thm*}[Existence of primitives]
    Let $f: \Omega \rightarrow \mathbb{C}$ be a function, $\Omega\subset\mathbb{C}$ open, connected.

    $f$ has a primitive if and only if, it is continuous and for any closed rectifiable path $\gamma$: 
    $$ \int_\gamma f(z)dz = 0 $$
\end{thm*}

\begin{note}
    This is a very useful property: to show that a function does not have a primitive, just show a counter example.
\end{note}

\begin{thm*}[Set of primitives]
    Let $f: \Omega \rightarrow \mathbb{C}$ be a function, $\Omega\subset\mathbb{C}$ open.

    If $g$ is a primitive of $f$, then $h$ is also a primitive of $f$ if and only if $h - g$ is constant.
\end{thm*}

\begin{thm*}[Integration by parts]
    Let $\gamma$ rectifiable on $\Omega$, open connected.
    
    
     $\forall f, g: \Omega \rightarrow \mathbb{C}$, holomorphic on $\Omega$ open, connected,

     $$ \int_\gamma f'g = [fg]_\gamma - \int_\gamma fg'$$
\end{thm*}

\newpage
\section{Third Session: Connected sets}
In this section, we characterize subsets of the complex plane (we refer to them simply as sets) that are "in one piece".

We present two alternative concepts: path-connectedness and connectedness. The latter is more abstract but more powerful. When the sets we consider are open, the two properties are equivalent.

\subsection*{Contents}
\begin{enumerate}
    \item \dots
        \begin{enumerate}
            \item \dots
        \end{enumerate}
\end{enumerate}

\subsection{Path-connected, connected}

\subsubsection{Path-connected}
We've already seen and used the definition of path connected but as a reminder:
\begin{defi}[Path-connected set]
   $A$ is path connected if any two points of $A$ can be joined by a path on $A$:

   $$\forall a,b\in A, \exists \gamma\in\mathcal{C}^0, \forall t\in[0,1], \gamma(t)\in A~\mathrm{ and}~\gamma(0) = a, \gamma(1) = b$$
\end{defi}

\subsubsection{Connected}
In order to define connectedness, let's introduce the concept of \emph{dilation}.

\begin{defi}[Dilation]
    A set $B$ is a dilation of $A$ if it the union of a collection of non-empty open disks whose centers are at the points of A:

    $$ B = \bigcup_{a\in A}{D(a, r_a)} $$
    where $\forall a\in A, r_a > 0$.
\end{defi}

We can now define connectedness:

\begin{defi}
    A set is connected if all of its dilations are path connected. A set that does not verify this property is disconnected.
\end{defi}

\begin{thm*}[Path-connected $\implies$ connected]
    Every path-connected set is connected.
\end{thm*}

\begin{proof}[Proof sketch $(\implies)$]
    Take $z, w \in B$ a dilation of $A$. There is $a, b \in A$ such that $z\in D_a$ and $w\in D_b$.
    As in the drawing, there is a path from $z$ to $a$ in $D_a$, and path from $b$ to $w$. Because $A$ is path-connected, there is a path from $a$ to $b$ in $A$. 
\end{proof}

\begin{thm*}[Open, connected $\implies$ path-connected]
    Every open, connected set is path-connected.
\end{thm*}

Let's quickly prove this theorem with a drawing:

\begin{proof}[Proof sketch $(\impliedby)$, not as pedagogical]
    Show that it is legitimate to write $A$ as a dilation of itself.
\end{proof}

Finally, we can deduce the following theorem:

\begin{thm*}[Open connected $\iff$ open, path-connected]
   An open set is connected if and only if it is path-connected. 
\end{thm*}

\subsubsection{Properties of path-connected/connected sets}

\begin{thm*}
    The following properties are shared by both connected and path-connected sets:
    \begin{description}
        \item[$\cap \mathcal{A} \neq \emptyset \implies \cup \mathcal{A}$ connected: ] if $\mathcal{A}$ is a collection of (path-)conected sets whose intersection $\cap \mathcal{A}$ is non-empty, then the union $\cup \mathcal{A}$ is (path-)connected.
        \item[$A \cap B = \emptyset$, with $A,B$ open $ \implies A\cup B$ disconnected] If $A$ and $B$ are two open, non-empty and disjoint sets, then their union is not (path-)connected.
    \end{description}
\end{thm*}

The following property only holds for connectedness:
\begin{thm*}[Closure of connected sets]
    The closure of a connected set is connected.
\end{thm*}

\begin{note}
    This is \emph{not} true for all path-connected sets. Refer to the textbook for a counter-example.
\end{note}

\subsection{Components}
\begin{defi}
    $B\subset A$ is a (path-)-connected component of $A$ if:

    \begin{itemize}
        \item $B$ is (path-)connected
        \item $B$ is maximal with respect to inclusion (i.e. if $B\subsetneq C\subset A$, then $C=A$)
    \end{itemize}
\end{defi}

\begin{thm*}
    The (path-)connected components of a non-empty set $A$ are a partition of $A$, i.e.:
    \begin{itemize}
        \item they are non-empty
        \item pairwise disjoint 
        \item union is $A$.
    \end{itemize}
\end{thm*}

\begin{thm*}
    A non-empty set is (path-)connected if and only if it has a single (path-)connected component.
\end{thm*}

Because connectedness and path-connectedness are equivalent when a set is open, we can show that:

\begin{thm*}
    The partitions of a non-empty open set into path-connected components and connected components are identical. All such components are open.
\end{thm*}

\begin{note}
    This result signifies that we can use "the" decomposition into (path-)connected components.
\end{note}

\subsection{Locally constant functions}
The following concept will be useful in the future.

\begin{defi}
    A function $f: A \rightarrow \mathbb{C}$ is locally constant if for any $a\in A$ there is a non-empty open disk $D$ centered on $a$ such that $f$ is constant on $A\cap D$.

    In other words, if:
    $$ \forall a \in A, \exists \varepsilon >0, \forall b \in A, |b-a|< \varepsilon \implies f(b) = f(a) $$
\end{defi}

\begin{thm*}
    A set $A$ is connected if and only if every locally constant function defined on $A$ is constant.
\end{thm*}

\newpage
\section{Fourth Session: Cauchy's integral theorem}
More so than the other chapters, this chapter is very condensed compared to the textbook. This is mainly because the textbook gives proofs for all the theorems presented.

We will skip the intermediate results and only present the results we need in subsequent chapters. Time-permitting, we can take a look at the different steps of the proof.

\begin{thm*}[Cauchy's integral theorem -- star-shaped version]
    Let $f: \Omega \rightarrow \mathbb{C}$ be a holomorphic function where $\Omega$ is an open star-shaped subset of $\mathbb{C}$.

    For any rectifiable closed path $\gamma$ of $\Omega$:

    $$ \int_\gamma f(z)dz = 0 $$
\end{thm*}

Recall the definition of star-shaped:

\begin{defi}[Star-shaped set]
    A open set $\Omega$ is star-shaped if there exist at least one point $c\in\Omega$ such that for all other points $z$ in $\Omega$, the segment $[c, z]$ is included in $\Omega$.
\end{defi}

\subsection*{Consequences of Cauchy's Integral Theorem}
CIT gives many interesting corollaries that are useful in practice.


\begin{thm*}[Cauchy's Integral Formula for Disks]
    Let $\Omega$ be an open subset of the complex plane. 

    Let $\gamma = c + r[C_+]$ be an oriented circle such that $B_f(c, r) \subset \Omega$.

    For any holomorphic function $f: \Omega \rightarrow \mathbb{C}$, 

    $$\forall z\in\mathcal{B}_f(c, r), f(z) = \frac{1}{2i\pi}\int_\gamma \frac{f(w)}{w - z}dw$$
\end{thm*}


\begin{thm*}
    The derivative of a holomorphic function if holomorphic.
\end{thm*}

\begin{proof}
    You'll prove this theorem during the problem session.
\end{proof}

\begin{note}
    We've already used this property in the past, but rest assured, there are no circular arguments.
\end{note}

\begin{thm*}[Morera's theorem -- characterisation of holomorphy]
    Let $\Omega$ be an open subset of the complex plane. 

    A function $f: \Omega \rightarrow \mathbb{C}$ is holomorphic if and only if:

    \begin{itemize}
        \item it is continuous
        \item locally, its line integrals along rectifiable closed paths are zero.
    \end{itemize}

    In other words, if and only if: for any $c\in\Omega$, there is an $r > 0$ such that $D(c, r) \in \Omega$ and for any rectifiable closed path $\gamma$ of $D(c,r)$,
    $$\int_\gamma f(z)dz = 0 $$
\end{thm*}

\begin{thm*}[Uniform limit of Holomorphic functions]
    Let $\Omega$ be an open subset of the complex plane. 

    If $f_n: \Omega \rightarrow \mathbb{C}$ a sequence of holomorphic functions converges loclly uniformly to a function $f: \Omega \rightarrow \mathbb{C}$, then $f$ is holomorphic.

    In other words, if $\forall c\in\Omega, \exists r > 0, D(c,r) \subset \Omega$ and 

    $$ \lim_{n\rightarrow \infty}\left\| f_n - f \right \|  = 0$$

\end{thm*}

Finally, a last useful theorem:

\begin{thm*}[Liouville's theorem]
    A bounded, holomorphic function defined on $\mathbb{C}$ (\emph{entire}) is constant.
\end{thm*}

Because this theorem is very useful (and to use the previous results) let's prove it step-by-step.

\begin{proof}
    To show that $f$ is constant, we show that $f' = 0$.
    \begin{description}
        \item[First, apply Cauchy's formula for disks to $f'$.]
            We get:
            $$ \forall z\in\mathbb{C}, f'(z) = \frac{1}{2i\pi} \int_\gamma\frac{f'(w)}{w-z}dw$$
            where $\gamma = z + r[C_+]$.
        \item[Second, integrate by parts to make $f$ appear.]
            We get:
            $$ \forall z\in\mathbb{C}, f(z) = \frac{1}{2i\pi} \int_\gamma\frac{f(w)}{(w-z)^2}dw$$
        \item[Third, show that $|f'| < \frac{K}{r}$ for all $r > 0$.] Use the M-L inequality.
        \item[Conclude.] With $r\rightarrow \infty$, $f'(z) = 0$. Because the zero function and $f$ are both primitives of $f'$, they differ by a constant (see Primitives chapter). Thus, $f$ is constant.
    \end{description}
\end{proof}


\newpage
\section{Fifth Session: the winding number ("le nombre venteux")}

Our goal in this chapter is to provide the necessary tools to allow us to generalize Cauchy's Integral Theorem to a global version.

The main idea is to properly define the inside and the outside of a path.

\subsection{Choices in arguments}

The goal of this section is to show how one can define "the argument" of a complex number, as a continuous function. 

The idea is as follows:
\begin{enumerate}
    \item Define a set-valued argument function (of all possible arguments).
    \item Define a choice of argument among these possibilities, a single-valued function.
    \item Define this choice with respect to a path.
\end{enumerate}

\begin{defi}[Argument function]
    The set-valued function $\mathrm{Arg}$ is defined on $\mathbb{C}^*$ by:
$$ \mathrm{Arg}(z) = \left \lbrace \theta \in \mathbb{R} \bigg\vert e^{i\theta} = \frac{z}{|z|}\right \rbrace$$
\end{defi}

A choice of the argument can be for example the Principal value of the Argument:

\begin{thm*}
    The principal value of the argument is the unique continuous function

    $$\mathrm{arg}: \mathbb{C} \setminus \mathbb{R}^{-} \rightarrow \mathbb{R}$$

    such that $$ \mathrm{arg} 1 = 0$$ 
\end{thm*}

\begin{note}
    Notice the \emph{cut} in $\mathbb{C}$ (classically along the negative real axis). This cut is unavoidable: there is no continuous choice of argument defined on $\mathbb{C}^*$.
\end{note}

However, in the special case of the argument along a path of $\mathbb{C}^*$, there is no restriction:

\begin{thm*}[Continuous choice of argument on a path]
    Let $a\in\mathbb{C}$ and $\gamma$ be a path of $\mathbb{C} \setminus \lbrace a \rbrace$.

    Let $\theta_0\in\mathbb{R}$ be a value of the argument of $\gamma(0) - a$, i.e. $\theta_0 \in \mathrm{Arg}(\gamma(0) - a)$.

    There is a unique function $\theta: [0,1] \rightarrow \mathbb{R}$ such that $\theta(0) = \theta_0$, which is a choice of $z \mapsto \mathrm{Arg}(z-a)$ on $\gamma$:

    $$\forall t \in [0,1], \theta(t) \in \mathrm{Arg}(\theta(t) - a) $$
\end{thm*}

We won't prove the theorem but let's just take look at what changes when on a path, thats makes a continuous choice possible.

\begin{proof}[Proof: what changes?]
We know the "history" of the path!
\end{proof}

\begin{note}
    Notice that the pointwise differences of two conitnuous choices of the argument along a path differ by a multiple of $2\pi$.
\end{note}

\begin{defi}[Variation of the argument]
    Let $a\in\mathbb{C}$ and $\gamma$ be a path of $\mathbb{C} \setminus \lbrace a \rbrace$.

    The variable of $z \mapsto \mathrm{Arg}(z-a)$ on $\gamma$ is defined as:

    $$[z \mapsto \mathrm{Arg}(z-a)]_\gamma = \theta(1) - \theta(0) $$

    This definition is unambiguous (proof in textbook).
\end{defi}

\subsection{Winding number \& Properties}
With these definitions in place, let's move on to the defintion of the winding number ("nombre venteux" ;)). 

Simply put, the winding number (or index) is a riguorous definition of a simple quantity: the number of times a path winds around a given point.

\begin{defi}[Winding number, index]
    Let $a\in\mathbb{C}$ and $\gamma$ be a path of $\mathbb{C} \setminus \lbrace a \rbrace$.
    
    The winding number (or index) of $\gamma$ around $a$ is the integer

    $$ \mathrm{ind}(\gamma, a) = \frac{1}{2\pi}[z \mapsto \mathrm{Arg}(z-a)]_\gamma $$
\end{defi}

Seen as a function of $\gamma, a$, we can show that $\gamma, a \mapsto \mathrm{Ind}(\gamma, a)$ is locally constant. That is: there exists a disk around $a$ and a "sleeve" (a sausage?) around $\gamma$ such that is by push $a$ or $gamma$ within these limits, their value does not change.

Recall the definition of locally constant:

\begin{defi}[Reminder: locally-constant function]
    A function $f: A \rightarrow \mathbb{C}$ is locally constant if for any $a\in A$ there is a non-empty open disk $D$ centered on $a$ such that $f$ is constant on $A\cap D$.

    In other words, if:
    $$ \forall a \in A, \exists \varepsilon >0, \forall b \in A, |b-a|< \varepsilon \implies f(b) = f(a) $$
\end{defi}

Let's formalize this:

\begin{thm*}[Ind is locally-constant]
    Let $a\in\mathbb{C}$ and $\gamma$ be a path of $\mathbb{C} \setminus \lbrace a \rbrace$.
   
    There is an $\varepsilon > 0$ such that, for any $b\in \mathbb{C}$ and closed path $\beta$, if 
    \begin{itemize}
        \item $|b - a| < \varepsilon$
        \item $\forall t \in [0,1], |\beta(t) - \gamma(t)| < \varepsilon $
    \end{itemize}

    then, $$\mathrm{ind}(\gamma, a) = \mathrm{ind}(\beta, b)$$
\end{thm*}

Finally, let's formalize an intuition that on the pieces of $\mathbb{C}$ that $\gamma$ cuts up, the winding number is constant. In other words:

\begin{thm*}[Ind constant on components]
    Let $\gamma$ be a closed path.

    The function
    $$ z \in \mathbb{C} \setminus \gamma([0,1]) \mapsto \mathrm{ind}(\gamma, z) $$
    is constant on each component of $\mathbb{C}\setminus \gamma([0,1])$.
\end{thm*}

\begin{thm*}{Winding number of unbounded components}
    Let $\gamma$ be a closed path.

    The function
    $$ z \in \mathbb{C} \setminus \gamma([0,1]) \mapsto \mathrm{ind}(\gamma, z) $$
    is $0$ on unbounded components.
\end{thm*}

\subsubsection{Interior, exterior \& simply connected sets}

Our goal here is to characterize sets that have no holes (on top of being in one piece). We'll then link this to index of a path, using the concepts of interior and exterior.

First, let's define:

\begin{defi}[Hole, Simply-connected]
    Let $\Omega\subset\mathbb{C}$ open.

    \begin{itemize}
        \item A hole of $\Omega$ is a bounded component of its complement $\mathbb{C} \setminus \Omega$.
        \item $\Omega$ is simply connected if it has no hole. In other words, if every component of its complement is unboounded. $\Omega$ is multiply connected otherwise.
    \end{itemize}
\end{defi}

\begin{example}[Simply connected but not connected]
    $$\Omega = \left\lbrace z \in \mathbb{C} \bigg\vert Re(z) < -1 ~\mathrm{or}~ Re(z) > 1 \right\rbrace$$
\end{example}

\begin{example}[Connected but not simply connected]
    $$\Omega = \mathbb{C} \setminus \left\lbrace 1, i, -i, -1, 0 \right\rbrace$$
\end{example}

In terms of paths, we should be able to draw a path around a hole. Thus a set is simply connected if any closed path that we draw has its "interior" in the set.

Before formalizing this intuition, let's define interior and exterior of a path.

\begin{defi}[Exterior of a closed path]
    The exterior of a closed path $\gamma$ is defined by:
    $$ \mathrm{Ext}\gamma = \left\lbrace z \in \mathbb{C} \setminus \gamma([0,1]) \bigg\vert  \mathrm{ind}(\gamma, z) = 0 \right\rbrace$$
\end{defi}

\begin{defi}[Interior of a closed path]
    The interior of a closed path $\gamma$ is defined by:
    $$ \mathrm{Int}\gamma = \mathbb{C} \setminus \left( \gamma([0,1]) \cup \mathrm{Ext}\gamma\right ) = \left\lbrace z \in \mathbb{C} \setminus \gamma([0,1]) \bigg\vert  \mathrm{ind}(\gamma, z) \neq 0 \right\rbrace$$
\end{defi}


\begin{thm*}[Index \& Simply Connected Sets]
    $\Omega\subset\mathbb{C}$ is simply connected if and only if the interior of any closed path $\gamma$ of $\Omega$ is included in $\Omega$:

    $$\forall z \in \mathbb{C}\setminus\gamma([0,1]), \mathrm{ind}(\gamma, z) \neq 0 \implies z \in \Omega $$

Alternatively, if and only if the complement of $\Omega$ is included in the exterior of $\gamma$:
$$\forall z \in \mathbb{C}\setminus\Omega, \mathrm{ind}(\gamma, z) = 0$$
\end{thm*}

\begin{example}
    Drawings with previous examples.
\end{example}

\begin{note}
    In the second example above, notice that we cannot always circle only one hole (they get infinitely close, one to another).
\end{note}

\subsection{Index as a line integral}
In order for there concepts to be useful, we will show that there is a relation between the index (or winding number) and the line integral.

Our goal is to define the variation of the argument on a path as a line integral. We are going to show that for a closed path $\gamma$ and $a\in\mathbb{C}\setminus\gamma([0,1])$:
$$\mathrm{ind}(\gamma, a) = \frac{1}{2i\pi}\int_\gamma\frac{dz}{z - a}$$

An idea behind this formula is that $z \mapsto \frac{1}{z - a}$ is the derivative of a logarithm function, which is pretty much a choice of argument (ignoring all the technical details).

First, we'll state the theorem, which we won't prove. However, we'll prove a necessary Lemma after, in order to practice.


\begin{thm*}[Index as a line integral]
    Let $a\in\mathbb{C}$ and $\gamma$ be a path of $\mathbb{C} \setminus \lbrace a \rbrace$.

    Then:

    $$ [z \mapsto \mathrm{Arg}(z - a)]_\gamma = \mathrm{Im}\left(\int_\gamma \frac{dz}{z-a}\right)$$

    If $\gamma$ is closed:
    $$\boxed{\mathrm{ind}(\gamma, a) = \frac{1}{2i\pi}\int_\gamma\frac{dz}{z-a}}$$
\end{thm*}

\begin{exo}[Lemma, for practice]
    Let $a\in\mathbb{C}$ and $\gamma$ be a path of $\mathbb{C} \setminus \lbrace a \rbrace$.

    For any $t\in[0,1]$, let $\gamma_t$ be the path such that for any $s\in[0,1]$, $\gamma_t(s) = \gamma(ts)$. 

    Let $\mu : [0,1] \rightarrow \mathbb{C}$ defined by:
    $$ \mu(t) = \int_{\gamma_t} \frac{dz}{z-a}$$ 

    Show that:
    $$\exists \lambda\in\mathbb{C}^*, \forall t \in [0,1], e^{\mu(t)} = \lambda \times (\gamma(t) -a) $$
\end{exo}

\newpage
\section{Sixth session: Cauchy's Integral Theorem  -- Global version}

Recall Cauchy's integral theorem for star-shaped open subsets of $\mathbb{C}$.

\begin{thm*}[Reminder: Cauchy's integral theorem -- star-shaped version]
    Let $f: \Omega \rightarrow \mathbb{C}$ be a holomorphic function where $\Omega$ is an open star-shaped subset of $\mathbb{C}$.

    For any rectifiable closed path $\gamma$ of $\Omega$:

    $$ \int_\gamma f(z)dz = 0 $$
\end{thm*}

Our goal here is to relax the star-shaped hypothesis; this will unlock additional properties.

We'll start by formulating the theorem. Then we'll provide a necessary tool: path sequences, an easy extension of closed paths. Then, we'll present some important corollaries of the global version of Cauchy's Integral Theorem.


\begin{thm*}[Cauchy's integral theorem -- global version]
    Let $f: \Omega \rightarrow \mathbb{C}$ be a holomorphic function where $\Omega$ is an open subset of $\mathbb{C}$.

    For any sequence of rectifiable closed paths $\gamma$ of $\Omega$ such that $\mathrm{Int}~\gamma \subset \Omega$:

    $$ \int_\gamma f(z)dz = 0 $$
\end{thm*}

\subsection*{Path sequences}
The extension from (closed) path to path sequence is straight-forward, so we'll go quickly. 

\begin{defi}
    The following definitions generalize easily:

    \begin{description}
        \item[Opposite \& Concatenation] The opposite of the path sequence $\gamma = (\gamma_1, \dots, \gamma_n)$ is the path sequence:
            $$\gamma^\leftarrow  = (\gamma_n^\leftarrow, \dots, \gamma_1^\leftarrow)$$
        \item[Image] The image of a path sequence $\gamma$ defined as above is:
            $$ \gamma([0,1]) = \bigcup_{k=1}^n{\gamma_k([0,1])} $$
        \item[Winding number] if $\gamma$ is a path sequence and $a\in\mathbb{C}\setminus\gamma([0,1])$,
            $$\mathrm{ind}(\gamma, a) = \sum_{k=1}^n\mathrm{ind}(\gamma_k, a)$$

        \item[Exterior] 
            $$\mathrm{Ext}\gamma = \left\lbrace z\in\mathbb{C}\setminus\gamma([0,1]) \vert \mathrm{ind}(\gamma, a) = 0 \right\rbrace$$
        \item[Interior] 
            $$\mathrm{Ext}\gamma = \left\lbrace z\in\mathbb{C}\setminus\gamma([0,1]) \vert \mathrm{ind}(\gamma, a) \neq 0 \right\rbrace$$
        \item[Length] 
            $$\ell(\gamma) = \sum_{k=1}^n \ell(\gamma_k)$$
        \item[Line integrals]
            $$\int_\gamma f(z)dz = \sum_{k=1}^n \int_{\gamma_k}f(z)dz$$
    \end{description}
\end{defi}

\subsection{Equivalent forms of Cauchy's Interal Theorem (global version)}


\subsubsection{Singularities \& Residue: Cauchy's Residue Theorem}

A singularity is an isolated point in $\mathbb{C}\setminus\Omega$. In particular it is not on the boundary of $\Omega$. It can be useful to know more about the nature of a singularity. We'll see more about that in the next session.

\begin{defi}[Singularity]
    Let $f: \Omega \rightarrow \mathbb{C}$ be a holomorphic function where $\Omega$ is an open subset of $\mathbb{C}$.
    
    $a\in\mathbb{C}\setminus\Omega$ is a singularity of $f$ if
    $$ \exists \varepsilon > 0 \vert \forall z\in\mathbb{C}, z \neq a, |z-a| < \varepsilon \implies z\in\Omega$$
    
\end{defi}
\begin{example}[$\star$]
    Show that $0$ is a singularity for $z \mapsto \frac{1}{z}$.
\end{example}

One way of characterizing a singularity could be to study the integral of $f$ "around" the singularity. To this end, we define:

\begin{defi}
    Let $f: \Omega \rightarrow \mathbb{C}$ be a holomorphic function where $\Omega$ is an open subset of $\mathbb{C}$.

    Let $a\in\mathbb{C}\setminus\Omega$ be a singularity of $f$.

    If $r>0$ is such that the only singularity in $\mathrm{Int}~\gamma$ is $a$, we define in the residure of $f$ at $a$ as:

    $$\mathrm{res}(f, a) = \frac{1}{2i\pi} \int_{a + r[C_+]} f(z)dz$$
\end{defi}

\begin{exo}[$\star$]
    \begin{enumerate}
        \item (Lemma) If the interior of $(\gamma, \mu^\leftarrow)$ in included in $\Omega$, then:
            $$\int_\gamma f(z)dz = \int_\mu f(z)dz$$
        \item Prove the independence of the residue from the choice of $r$.
    \end{enumerate}
\end{exo}

\begin{exo}[$\star$]
    \begin{enumerate}
        \item Calculate the residue of $z \mapsto \frac{1}{z-a}$.
        \item Calculate the residue of $z \mapsto (z-a)^n$.
    \end{enumerate}
\end{exo}

\begin{thm*}[Cauchy's Residue Theorem]
    Let $\Omega$ be an open subset of $\mathbb{C}$ and let $f: \Omega \rightarrow \mathbb{C}$ be a holomorphic function.

    Let $\gamma$ be a sequence of rectifiable closed paths of $\Omega$ and $a\in\Omega\setminus\gamma([0,1])$.

    If $A$ is a finite set of isolated singularities of $f$ such that $\mathrm{Int}~\gamma \subset \Omega \cup A$ then:
    $$\int_\gamma \frac{f(z)}{z-a}dz = 2i\pi \times \sum_{a\in A}\mathrm{ind}(\gamma, a) \times \mathrm{res}(f,a)$$
\end{thm*}

\begin{note}
    Notice that if $A = \emptyset$, we obtain Cauchy's Integral Theorem.
\end{note}

\subsubsection{Cauchy's Integral Formula}
\begin{thm*}[Cauchy's Integral Formula]
    Let $\Omega$ be an open subset of $\mathbb{C}$ and let $f: \Omega \rightarrow \mathbb{C}$ be a holomorphic function.

    Let $\gamma$ be a sequence of rectifiable closed paths of $\Omega$ and $a\in\Omega\setminus\gamma([0,1])$.

    If $\mathrm{Int}~\gamma\subset\Omega$ then:
    $$\int_\gamma \frac{f(z)}{z-a}dz = 2i\pi \times \mathrm{ind}(\gamma, a) \times f(a)$$
\end{thm*}

\newpage

\section{Seventh session: Power series}

A power series is the generalization of the polynomial to an infinity of terms, written:
$$\sum_{n=0}^{+\infty}a_n(z-c)^n$$

\subsection{Definition \& Fundamental properties}

\begin{defi}
    The radius of convergence for the above power series is the unique $r$ such that:
    \begin{itemize}
        \item the series converges if $|x - c| < r$ 
        \item the series diverges is $|x - c| > r$.
    \end{itemize}

    In other words, it is the inverse of the growth ratio $\sigma$ of the series, such that:
    $$\exists m \in \mathbb{N}, \forall n \in \mathbb{N} \vert n \geq m \implies |a_n| \leq \sigma^n $$
\end{defi}

In pratice to calculate the radius of convergence, find the smallest $\sigma > 0$ such that "apcr" $$|a_n|^n \leq \sigma^n$$

\begin{note}
    Careful, at the radius of convergence, anything can happen!
\end{note}

\begin{example}
    [$\star$]
    $a_n = 2^n$, $a_n = \left(\frac{1}{2}\right)^n$, \dots
\end{example}

\begin{thm*}
    [Multiplication of power series]
    $$R(a_nb_n) \geq R(a_n)R(b_n)$$

    Equality for polynomial coefficients, i.e. $a_n \sim n^p$.
\end{thm*}
\begin{proof}
    $\sigma_{ab} \leq \sigma_a\sigma_b$
\end{proof}

\subsection{Convergence}
Let's take a look at what happens within a power series' disk of convergence.

\begin{thm*}
    [Locally Normal Convergence]
    The convergence of the power series above in its open disk of convergence $D(c, r)$ is locally normal:
    
    for any $z\in D(c,r)$, there exists an open neighborhood $U$ of $z$ in $D(c, r)$ such that

    $$\exists \kappa > 0, \forall z \in U, \sum_{n=0}^{+\infty}|a_n(z-c)^n| \leq \kappa$$
\end{thm*}


    Equivalently, for every compact subset $K$ of $D(c,r)$,
    $$\exists \kappa > 0, \forall z \in K, \sum_{n=0}^{+\infty}|a_n(z-c)^n| \leq \kappa$$

\end{document}
